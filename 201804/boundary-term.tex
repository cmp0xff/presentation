%% UzK - A BEAMER THEME FOR THE UNIVERSITY OF COLOGNE
%% http://solstice.github.com/uzk-theme/

\documentclass[9pt]{beamer}

\usepackage{fontspec}

% Multilingual support
\usepackage{polyglossia}

% more symbols
\usepackage{textcomp}

% fontspec, realscripts, metalogo fot XeLaTeX
\usepackage{xltxtra}
	% Unicode fonts
	\setmainfont{CMU Serif}
	\setsansfont{CMU Sans Serif}
	\setmonofont{CMU Typewriter Text}
\usepackage{amsfonts}
\usepackage{unicode-math}
	\setmathfont{Latin Modern Math} % default
	\setmathfont{Latin Modern Math}[range={
		"1D608-"1D63B},	% Italic sans serif, Latin, uppercase
		sans-style=italic]
	\setmathfont{Latin Modern Math}[range={
		"1D6E2-"1D6FA,	% Italic Greek, uppercase
		"1D6FC-"1D71B},	% Italic Greek, lowercase
		math-style=ISO]
	\setmathfont{Latin Modern Math}[range={
		"00391-"003A9,	% Upright Greek, uppercase
		"003B1-"003F5,	% Upright Greek, lowercase
		"1D6A8-"1D6E1},	% Bold Greek
		sans-style=upright]
	\setmathfont{Asana Math}[range={
		\mathbin}] %\mathord
	\setmathfont{STIX Math}[range={
		"02609}] % ☉
	\setmathfont{XITS Math}[range={
		"1D4B6-"1D4CF}] % Script, Latin, lowercase
	%\setmathfont{⟨font name⟩}[range=⟨unicode range⟩,⟨font features⟩]

\usepackage{xeCJK}
\setmainlanguage{english}
\setotherlanguage{ngerman}

\usefonttheme{professionalfonts}
%  don't change fonts inside beamer

% AMS--related
\usepackage{amsmath,amssymb}

% ':=' as \coloneqq
\usepackage{mathtools}
\usepackage{siunitx}
\usepackage{cleveref}

% some unicode characters
% ≙ for equal with hat


% Mathematical constants
\newcommand{\ii}{{\Bbbi}}
\newcommand{\ee}{{\Bbbe}}
\newcommand{\pp}{{\Bbbpi}}

% Math operators
\DeclareMathOperator{\arcsinh}{arcsinh}
\DeclareMathOperator{\arccosh}{arccosh}
\DeclareMathOperator{\arctanh}{arctanh}
\DeclareMathOperator{\arccoth}{arccoth}
\DeclareMathOperator{\arcctgh}{arcctgh}
\DeclareMathOperator{\arcsech}{arcsech}
\DeclareMathOperator{\arccsch}{arccsch}

\DeclareMathOperator{\BesselJ}{J}
\DeclareMathOperator{\BesselY}{Y}
\DeclareMathOperator{\BesselF}{F}
\DeclareMathOperator{\BesselG}{G}
\DeclareMathOperator{\BesselI}{I}
\DeclareMathOperator{\BesselK}{K}
\DeclareMathOperator{\BesselL}{L}

\DeclareMathOperator{\sgn}{sgn}
\DeclareMathOperator{\grad}{grad}
\DeclareMathOperator{\curl}{curl}
\DeclareMathOperator{\rot}{rot}
\DeclareMathOperator{\opdiv}{div}
\DeclareMathOperator{\opdeg}{deg}

\DeclareMathOperator{\sech}{sech}
\DeclareMathOperator{\csch}{csch}

\DeclareMathOperator{\diag}{diag}
\DeclareMathOperator{\tr}{tr}
\DeclareMathOperator{\rank}{rank}

\DeclareMathOperator{\ad}{ad}

\DeclareMathOperator{\expi}{expi}

% Differentials
\newcommand{\DD}{\BbbD}
\newcommand{\dd}{\Bbbd}
\newcommand{\dva}{\mupdelta} % no better way?!
\newcommand{\Dva}{\mupDelta}

% Equal marks
\newcommand{\eeq}{{\overset{!}{=}}}
\newcommand{\lls}{{\overset{!}{<}}}
\newcommand{\ggt}{{\overset{!}{>}}}
\newcommand{\lle}{{\overset{!}{\le}}}
\newcommand{\gge}{{\overset{!}{\ge}}}

% Group and Algebras
\newcommand{\SO}{\msansS\msansO}
\newcommand{\SU}{\msansS\msansU}
\newcommand{\so}{\mfraks\mfrako}
\newcommand{\su}{\mfraks\mfraku}
% Bracket-like
\newcommand{\rbr}[1]{{\left(#1\right)}}
\newcommand{\sbr}[1]{{\left[#1\right]}}
\newcommand{\cbr}[1]{{\left\{#1\right\}}}
\newcommand{\abr}[1]{{\left<#1\right>}}
\newcommand{\vbr}[1]{{\left|#1\right|}}
\newcommand{\dvbr}[1]{{\left\|#1\right\|}}
\newcommand{\fat}[2]{{\left.#1\right|_{#2}}}
% Functions; note the space between the name and the bracket!
\newcommand{\rfun}[2]{{#1}\mathopen{}\left(#2\right)\mathclose{}}
\newcommand{\sfun}[2]{{#1}\mathopen{}\left[#2\right]\mathclose{}}
\newcommand{\cfun}[2]{{#1}\mathopen{}\left\{#2\right\}\mathclose{}}
\newcommand{\afun}[2]{{#1}\mathopen{}\left<#2\right>\mathclose{}}
\newcommand{\vfun}[2]{{#1}\mathopen{}\left|#2\right|\mathclose{}}
% Fraction-like
\newcommand{\frde}[2]{{\frac{\dd{#1}}{\dd{#2}}}}
\newcommand{\frDe}[2]{{\frac{\DD{#1}}{\DD{#2}}}}
\newcommand{\frpa}[2]{{\frac{\partial{#1}}{\partial{#2}}}}
\newcommand{\frdva}[2]{{\frac{\dva{#1}}{\dva{#2}}}}

% overline-like marks
\newcommand{\ol}[1]{{\overline{{#1}}}}
\newcommand{\ul}[1]{{\underline{{#1}}}}
\newcommand{\tld}[1]{{\widetilde{{#1}}}}
\newcommand{\ora}[1]{{\overrightarrow{#1}}}
\newcommand{\ola}[1]{{\overleftarrow{#1}}}
\newcommand{\td}[1]{{\widetilde{#1}}}
\newcommand{\what}[1]{{\widehat{#1}}}
%\newcommand{\prm}{{\symbol{"2032}}}

% Physical constants and parameters
\newcommand{\lc}{\mitsansc} % speed of light in vacuum
\newcommand{\bk}{\mitsansk} % Boltzmann's constant
\newcommand{\phs}{\hslash} % reduced Planck constant
\newcommand{\ph}{\Planckconst} % Planck constant

\newcommand{\nG}{\mitsansG} % Newton's constant
\newcommand{\aN}{\mitsansN} % Avogadro number
\newcommand{\ec}{\mitsanse} % unit electric charge

\newcommand{\gR}{\mitsansR} % gas constant

\newcommand{\plm}{m_\text{P}} % Planck mass
\newcommand{\pll}{l_\text{P}} % Planck length
\newcommand{\plt}{t_\text{P}} % Planck time

\newcommand{\hH}{\mitsansH} % Hubble parameter H
\newcommand{\hh}{\mitsansh} % Hubble parameter h
\newcommand{\dq}{\mitsansq} % Deceleration parameter q

\newcommand{\apE}{\alpha_\text{E}} % EM fine struct const
\newcommand{\apG}{\alpha_\text{G}} % Grav fine struct const

% Common symbols
\newcommand{\Ld}{\mscrL} % Lagrangian density
\newcommand{\fp}{p_\text{F}} % Fermi momentum
\newcommand{\fE}{\mscrE_\text{F}} % Fermi energy

% Others
\newcommand{\rSch}{R_\text{S}} % Schwarzschild radius

\newcommand{\fHor}{{\mscrh^+}} % future horizon
\newcommand{\pHor}{{\mscrh^-}} % past horizon

% Chemical elements
\usepackage[version=4]{mhchem}


% siunitx
% Astronomy
\DeclareSIUnit\parsec{pc}
\DeclareSIUnit\lightyear{ly}

\usepackage[%style=numeric,
			firstinits=true,
			isbn = false, doi = false,
			backend=biber]{biblatex}
%\addbibresource{./boundary-term.bib}
\AtEveryCitekey{
		\clearfield{issn}
		\clearfield{month}\clearfield{url}
	\ifentrytype{book}{
		\clearfield{publisher}
 	}{
		\clearfield{title}
	}
}

%% Falls Anzeige der \sections, \subsections etc. gewuenscht, kann zB.
%% das infolines theme geladen werden. Wichtig ist jedoch, dass andere
%% Themes _vor_ dem UzK-Theme geladen werden.
%\useoutertheme{infolines}

%% Falls keine der Optionen zur Bestimmung der Fusszeile uebergeben werden    %%
%% werden alle Fakultaetsfarben verwendet. ---------------------------------- %%
\usetheme[%
%wiso,        %% Wiso-Fakultaet
%jura,        %% Rechtswissenschaftliche Fakultaet
%medizin,     %% Medizinische Fakultaet
%philo,       %% Philosophische Fakultaet
%matnat,      %% Mathematisch-Naturwissenschaftliche Fakultaet
%human,       %% Humanwissenschaftliche Fakultaet
%verw,        %% Universitaetsverwaltung
%nav,         %% Schaltet die Navigationssymbole ein
latexfonts,  %% Verwendet die latex-beamer-Standardschrift
%colorful,    %% Farbige Balken im infolines-Theme
%squares,     %% Aufzaehlungspunkte rechteckig
%nologo,      %% Kein Logo im Seitenhintergrund
]{UzK}

\title{Boundary Actions in Geometrodynamics}
%\subtitle{Integrability, Self-adjointness and Semi-classical Wave Packets}

%\date{March 23, 2018}
\date{April, 2018}

\author[%Andrianov \and Lan \and Novikov \and \underline{Wang}
Wang]{
	%Alexander A. Andrianov\inst{1,4} %\thanks{a.andrianov@spbu.ru}
	%\and
	%Chen Lan\inst{2} %\thanks{stlanchen@yandex.ru}
	%\and
	%Oleg O. Novikov\inst{1} %\thanks{o.novikov@spbu.ru}
	%\and 
	%\underline{Yi-Fan Wang}\inst{3}} %\thanks{yfwang@thp.uni-koeln.de}
	Yi-Fan Wang\inst{3}}

\institute[%SPBU \and ELI-ALPS \and UzK \and UB
UzK]{
%\inst{1} Saint-Petersburg State University,
%Ulyanovskaya str. 1, Petrodvorets, Sankt-Petersburg 198504, Russland
%\and
%\inst{2}
%ELI-ALPS Research Institute,
%Budapesti út 5, H-67228 Szeged, Ungarn
%\and
\inst{3}
Institut für Theoretische Physik, Universität zu Köln,
Zülpicher Straße 77, D-50937 Köln, Deutschland
%\and
%\inst{4}
%Institut de Ciències del Cosmos, Universitat de Barcelona, Martí i Franquès 1, 
%E-08028 Barcelona, Spanien
}

\AtBeginSection[]
{
 \begin{frame}<beamer>
 \frametitle{Outline}
 \tableofcontents[currentsection]
 \end{frame}
}

\begin{document}

\begin{frame}%[titlepage]
  \titlepage
\end{frame}

\begin{frame}{Outline}
  \tableofcontents
  % You might wish to add the option [pausesections]
\end{frame}


\section{Introduction}

\begin{frame}%
{Introduction}%
{Quintessence and phantom Liouville field}
\begin{itemize}
\item Observed accelerated expansion can be explained by a cosmological 
constant%\footfullcite{COPELAND2006,Bamba2012} as stationary Dark Energy, but 
its 
origin has yet to be understood.

\item Dynamical Dark Energy has been modeled by
quintessence%\footfullcite{Caldwell1998} and phantom\footfullcite{Caldwell2002} 
matter, with barotropic index\footnote{Barotropic index is the $w$ in equation
of state $\rho = wp$.} $w > -1$ and $w < -1$, respectively.

\item They can be realised by minimally-coupled real scalar fields with $\mscrl 
= \pm 1$\footnote{The signature of metric is mostly positive.}
\begin{align}
\mscrS = \int\dd^4 x\,\sqrt{-g}\,\cbr{-\mscrl \frac{g^{\mu\nu}}{2}
\rbr{\partial_\mu\phi}\rbr{\partial_\nu\phi} - \rfun{\mscrV}{\phi}}.
\end{align}

\item $\rfun{\mscrV}{\phi} = V\ee^{\lambda\phi}$, $\lambda, V\in \BbbR$ is of
interest: Liouville field%\footfullcite{NAKAYAMA2004}.
\end{itemize}
\end{frame}

\begin{frame}%
{Introduction}%
{Friedmann--Lemaître model}
\begin{itemize}
\item Assume flat Robertson--Walker metric for a homog.\ and isotr.\ model
\begin{align}
g_{\mu\nu}\,\dd x^\mu\,\dd x^\nu = -\rfun{N^2}{t}\,\dd t^2 + \varkappa 
\ee^{2\rfun{\alpha}{t}}\,\dd\Omega_{3\text{F}}^2
\end{align}
w/ $\varkappa \coloneqq 8\pp\nG$, $\dd\Omega_{3\text{F}}^2 \coloneqq 
\dd\chi^2+\chi^2\rbr{\dd\theta^2+\sin^2\theta\,\dd\varphi^2}$, $N$
lapse function.

\item Combined with the Liouville field, the total action reads
$\mscrS \coloneqq \mscrS_{\text{EH}} + \mscrS_\text{GHY} + \mscrS_\text{L}
= \int\dd\Omega_{3\text{F}}^2\int\dd t\,L$, where
%in which the effective Lagrangian reads
\begin{align}
%S_{\text{Einstein--Hilbert}} + S_\text{Gibbons--Hawking--York} + 
%S_\text{Liouville}
L&\coloneqq \varkappa^{3/2}N\ee^{3\alpha}
\rbr{-\frac{3}{\varkappa}\frac{\dot{\alpha}^2}{N^2}
+\mscrl\frac{\dot{\phi}^2}{2N^2}-V\ee^{\lambda\phi}},
\label{eq:eff-lag-10}
\end{align}
in which dot means $\dd/\dd t$.

\item The model turns out to be integrable, both classically and 
quantum-mechanically, enabling one to study its full physical properties, e.g.\
the relation between its classical and quantum theory.
\end{itemize}
\end{frame}

\section{Classical model and the implicit trajectories}

\subsection{Lagrangian formalism}

\begin{frame}%
{Decoupling the variables}%
{Via \emph{rescaled} special orthogonal transformation}
\begin{itemize}
\item Setting $\overline{N} \coloneqq N\ee^{-3\alpha}$, \cref{eq:eff-lag-10}
%the effective Lagrangian
transforms to
\begin{align}
L = \varkappa^{3/2}\overline{N}
\rbr{-\frac{3}{\varkappa}\frac{\dot{\alpha}^2}{\overline{N}^2}
+\mscrl\frac{\dot{\phi}^2}{2\overline{N}^2} - V\ee^{\lambda\phi+6\alpha}}
\end{align}

\item Defining $\Delta \coloneqq \lambda^2 - 6\mscrl\varkappa$,
$\mscrs \coloneqq \sgn \Delta$ and
$g \coloneqq \mscrs \sqrt{\vbr{\Delta}} \equiv \mscrs\sqrt{\mscrs\Delta}$,
the \emph{rescaled} special orthogonal transformation
\begin{align}
\begin{pmatrix}
\alpha \\ \phi
\end{pmatrix} = \frac{\mscrs}{g}
\begin{pmatrix}
\lambda & -\mscrl\varkappa \\
-6 & \lambda
\end{pmatrix}
\begin{pmatrix}
\mscrs_\beta \beta \\ \mscrs_\chi \chi
\end{pmatrix}\qquad\text{where } \mscrs_\beta, \mscrs_\chi = \pm 1
\end{align}
gives the decoupled Lagrangian
\begin{align}
L = \varkappa^{3/2}\overline{N}
\rbr{-\mscrs\frac{3}{\varkappa}\frac{\dot{\beta}^2}{\overline{N}^2}
+\mscrl\mscrs\frac{\dot{\chi}^2}{2\overline{N}^2}
-V\ee^{\mscrs_\chi g\chi}}.
\label{eq:lagrangian-decoupled}
\end{align}

\item
The Euler--Lagrange equations w.r.t.\ $\overline{N}$, $\beta$ and
$\chi$ will be called the trsfed.\ 1st, 2nd Friedmann eqs.\ and the
Klein--Gordon eq., respectively.
\end{itemize}
\end{frame}

\begin{frame}%
{Implicit integration}%
{General integral for $p_\beta \neq 0$}
\begin{itemize}
\item Since $\beta$ is cyclic in \cref{eq:lagrangian-decoupled},
the trsfed.\ 2nd Friedmann eq.\ can be integrated%

\item For $p_\beta \neq 0$, fixing the \emph{implicit gauge}
$\overline{N} = -6\mscrs\sqrt{\varkappa}\dot{\beta}/p_\beta$, the trsfed.\
1st Friedmann equation \alert{can be integrated to get the trajectory}

in which $\mscrv \coloneqq \sgn V$, $\rbr{\mathrm{sgn},\mathrm{sgn}}$ means 
$\rbr{\mscrl,\mscrs\mscrv}$, and

\end{itemize}
\end{frame}

\begin{frame}%
{Implicit integration}%
{Specific integral for $p_\beta = 0$}
\begin{itemize}
\item For $p_\beta = 0$, integrating the transformed second Friedmann 
equation yields $\beta \equiv \beta_0$ or $\phi-\phi_0 =
-\mscrl\lambda\alpha/\varkappa$, which is the well-known power-law special
solution%\footfullcite[For instance,][ch.\ 3]{Liddle2000}.

\item Further integrating the transformed first Friedmann equation demands 
$\rbr{+,-}$ or $\rbr{-,+}$ to guarantee $\overline{N} > 0$.

\end{itemize}
\end{frame}




\begin{frame}%
{Trajectories for quintessence model $\rbr{+,+}$}%
{$\sech$, with $\beta_0 = 0$, $\vbr{V} = \varkappa^{-2}$ and
$p_\beta^2 = \varkappa^2\sqrt{\vbr{V}}$; varying $\lambda$}
%\includegraphics[width=\textwidth]{../plots.nb/sech_lamb.pdf}
\begin{itemize}
	\item has two asymptotes $\chi \propto \pm \beta$
\end{itemize}
\end{frame}

\begin{frame}%
{Trajectories for quintessence model $\rbr{+,+}$}%
{$\sech$, with $\beta_0 = 0$, $\vbr{V} = \varkappa^{-2}$ and
$\lambda^2 = 3\varkappa$; varying $p_\beta$}
%\includegraphics[width=\textwidth]{../plots.nb/sech_pbet.pdf}
\begin{itemize}
	\item has two asymptotes $\chi \propto \pm \beta$
\end{itemize}
\end{frame}

\begin{frame}%
{Trajectories for quintessence model $\rbr{+,+}$}%
{$\sech$, with $\beta_0 = 0$, $\lambda^2 = 3\varkappa$ and
$p_\beta^2 = \varkappa^2\sqrt{\vbr{V}}$; varying $V$}
%\includegraphics[width=\textwidth]{../plots.nb/sech_Vsqr.pdf}
\begin{itemize}
	\item has two asymptotes $\chi \propto \pm \beta$
\end{itemize}
\end{frame}

% \begin{frame}%
% {Trajectories for quintessence model $\rbr{+,+}$}%
% {$\sech$, with $\beta_0 = 0$, $\lambda^2 = 3\varkappa$ and
% $p_\beta^2 = \varkappa^2\sqrt{\vbr{V}}$; varying $\varkappa$}
% %\includegraphics[width=\textwidth]{../plots.nb/sech_vark.pdf}
% \begin{itemize}
% 	\item has two asymptotes $\chi \propto \pm \beta$
% \end{itemize}
% \end{frame}


\begin{frame}%
{Trajectories for quintessence model $\rbr{+,-}$}%
{$\csch$, with $\beta_0 = 0$, $\vbr{V} = \varkappa^{-2}$ and
$p_\beta^2 = \varkappa^2\sqrt{\vbr{V}}$; varying $\lambda$}
%\includegraphics[width=\textwidth]{../plots.nb/csch_lamb_l.pdf}
\begin{itemize}
	\item contains two distinct solutions, separated by $\beta = 0$
	\item has three asymptotes $\chi \propto \pm \beta$ and $\beta = 0$
\end{itemize}
\end{frame}

\begin{frame}%
{Trajectories for quintessence model $\rbr{+,-}$: $\csch$}%
{$\csch$, with $\beta_0 = 0$, $\vbr{V} = \varkappa^{-2}$ and
$p_\beta^2 = \varkappa^2\sqrt{\vbr{V}}$; varying $\lambda$}
%\includegraphics[width=\textwidth]{../plots.nb/csch_lamb_r.pdf}
\begin{itemize}
	\item contains two distinct solutions, separated by $\beta = 0$
	\item has three asymptotes $\chi \propto \pm \beta$ and $\beta = 0$
\end{itemize}
\end{frame}

\begin{frame}%
{Trajectories for quintessence model $\rbr{+,-}$: $\csch$}%
{$\csch$, with $V = \varkappa^{-2}$ and
$\lambda^2 = \varkappa$; varying $p_\beta$}
%\includegraphics[width=\textwidth]{../plots.nb/csch_pbet_l.pdf}
\begin{itemize}
	\item contains two distinct solutions, separated by $\beta = 0$
	\item has three asymptotes $\chi \propto \pm \beta$ and $\beta = 0$
\end{itemize}
\end{frame}

\begin{frame}%
{Trajectories for phantom model $\rbr{-,+}$}%
{$\csc$, with $V = \varkappa^{-2}$ and
$p_\beta^2 = 3\varkappa^2\sqrt{\vbr{V}}$}
%\includegraphics[width=\textwidth]{../plots.nb/csc_lamb_l.pdf}
%\begin{columns}
%\begin{column}{.55\textwidth}
\begin{itemize}
	\item contains infinite distinct solutions
	\item has infinite parallel asymptotes $\beta \propto \rbr{n+1/2}\pp $
\end{itemize}
%\end{column}
%\begin{column}{.4\textwidth}
%\begin{itemize}
%\item is $\beta$-even for $\beta_0 = 0$
%\end{itemize}
%\end{column}
%\end{columns}
\end{frame}

\begin{frame}%
{Trajectories for phantom model $\rbr{-,+}$}%
{$\csc$, with $V = \varkappa^{-2}$ and
$p_\beta^2 = 3\varkappa^2\sqrt{\vbr{V}}$}
%\includegraphics[width=\textwidth]{../plots.nb/csc_lamb_r.pdf}
%\begin{columns}
%\begin{column}{.55\textwidth}
\begin{itemize}
	\item contains infinite distinct solutions
	\item has infinite parallel asymptotes $\beta \propto \rbr{n+1/2}\pp $
\end{itemize}
%\end{column}
%\begin{column}{.4\textwidth}
%\begin{itemize}
%\item is $\beta$-even for $\beta_0 = 0$
%\end{itemize}
%\end{column}
%\end{columns}
\end{frame}

\begin{frame}%
{Trajectories for phantom model $\rbr{-,+}$}%
{$\csc$, with $V = \varkappa^{-2}$ and
$\lambda^2 = 3\varkappa$; varying $p_\beta$}
%\includegraphics[width=\textwidth]{../plots.nb/csc_pbet_l.pdf}
%\begin{columns}
%\begin{column}{.55\textwidth}
\begin{itemize}
	\item contains infinite distinct solutions
	\item has infinite parallel asymptotes $\beta \propto \rbr{n+1/2}\pp $
\end{itemize}
%\end{column}
%\begin{column}{.4\textwidth}
%\begin{itemize}
%\item is $\beta$-even for $\beta_0 = 0$
%\end{itemize}
%\end{column}
%\end{columns}
\end{frame}

\begin{frame}%
{Integration}%
{Further discussions}
\begin{itemize}
%\item The integrals are consistent with the trsfed.\ Klein--Gordon equation.

\item The integral for $\rbr{-,-}$ is not real.

\item The trajectories can be parametrised by $\beta$, inspiring recognising
$\beta$ as a `time variable'.

\item The implicit integration enables one to compare trajectories with wave 
functions, see below.
\end{itemize}
\end{frame}

\section{Quantised model and the wave packets}

\subsection{Canonical formalism and Dirac quantisation}

\begin{frame}%
{Dirac quantisation and the mode functions}%
%{233}
\begin{itemize}
\item
The primary Hamiltonian and the Hamiltonian
constraint%\footfullcite{Gitman1990,Rothe2010} reads
\begin{align}
H^\text{p} &= \overline{N}H_\perp + v^{\overline{N}} p_{\overline{N}},
\\
H_\perp &\coloneqq -\mscrs\frac{p_\beta^2}{12\varkappa^{1/2}}
+\mscrl\mscrs\frac{p_\chi^2}{2\varkappa^{3/2}}
+\varkappa^{3/2}V\ee^{g\mscrs_\chi\chi}.
\end{align}
\item
Applying the Dirac quantisation rules with Laplace--Beltrami
operator%\footfullcite[ch.\ 8]{Kiefer2012}, one 
gets the minisuperspace Wheeler--DeWitt equation with $\rbr{\beta,\chi}$
%$\widehat{H}_\perp \rfun{\Psi}{\beta,\chi} = 0$,
\begin{align}
% -\mscrs \frac{12\varkappa^{1/2}}{\hslash^2}\widehat{H}_\perp 
% \Psi \eqqcolon -\partial_\beta^2\Psi - \BbbD\Psi,\qquad
% \BbbD = -\mscrl \frac{6}{\varkappa}\partial_\chi^2 +
% \mscrs \frac{12\varkappa^2 V \ee^{g\mscrs_\chi\chi}}{\hslash^2}.
0 &= \widehat{H}_\perp \rfun{\Psi}{\beta,\chi} \coloneqq
\rbr{\mscrs\frac{\hslash^2}{12\varkappa^{1/2}}\partial_{\beta}^2
-\mscrl\mscrs\frac{\hslash^2}{2\varkappa^{3/2}}\partial_{\chi}^2
+\varkappa^{3/2}V\ee^{g\mscrs_\chi\chi}}\Psi.
\label{eq:WDW-10}
\end{align}
\item
\Cref{eq:WDW-10} is KG-like, hyperbolic for $\mscrl = +1$ and 
\emph{elliptic} for $\mscrl = -1$.
\end{itemize}
\end{frame}

\begin{frame}%
{Separation of the variables and mode functions}%
%{233}
\begin{itemize}
\item
Writing $\rfun{\Psi}{\beta,\chi} = \rfun{\varphi}{\beta}
\rfun{\psi}{\chi}$, \cref{eq:WDW-10} can be separated into

\begin{align}
\partial_\beta^2\rfun{\varphi}{\beta} &= k_\beta^2\rfun{\varphi}{\beta};\\
\BbbD \rfun{\psi}{\chi} &= k_\beta^2 \rfun{\psi}{\chi},\qquad
\BbbD \coloneqq
-\mscrl \frac{6}{\varkappa} \partial_\chi^2
+ \mscrs\mscrv \frac{12\varkappa^2\vbr{V}}{\hslash^2}.
\label{eq:separated-10}
\end{align}
\item \Cref{eq:separated-10}
turns out to be Besselian, and the mode functions are
\begin{align}
\rfun{\Psi_\nu}{\beta, \chi} \coloneqq
\sum_{i=1}^2 c_i \rfun{\varphi_{\nu}^{(i)}}{\gamma}
\sum_{j=1}^2 a_j\rfun{\mathrm{B}_{\nu}^{(j)}}{\sigma},\quad
\nu \ge 0; \\
\nu \coloneqq \sqrt{\frac{2\varkappa}{3}}\frac{1}{g}\,k_\beta,\quad
\gamma \coloneqq \sqrt{\frac{3}{2\varkappa}}\frac{g}{1}\,\beta,\qquad
\sigma^2 \coloneqq 
\frac{8\varkappa^3\vbr{V}\ee^{g\mscrs_\chi\chi}}{\hslash^2 g^2},
\label{eq:trsf-quantum-sigma}
\\
\begin{aligned}
\rfun{{}_{\rbr{+,+}}\mathrm{B}^{(i)}_{\nu}}{\sigma} &\coloneqq
\mathrm{K}\text{ and }\rfun{\mathrm{I}_{\ii\nu}}{\sigma},\quad&
\rfun{{}_{\rbr{+,-}}\mathrm{B}^{(i)}_{\nu}}{\sigma} &\coloneqq
\mathrm{F}\text{ and }\rfun{\mathrm{G}_{\ii\nu}}{\sigma},\nonumber \\
\rfun{{}_{\rbr{-,+}}\mathrm{B}^{(i)}_{\nu}}{\sigma} &\coloneqq
\mathrm{J}\text{ and }\rfun{\mathrm{Y}_{\nu}}{\sigma},\quad&
\rfun{{}_{\rbr{-,-}}\mathrm{B}^{(i)}_{\nu}}{\sigma} &\coloneqq
\mathrm{K}\text{ and }\rfun{\mathrm{I}_{\nu}}{\sigma}.
\end{aligned}
\end{align}
\item Adapted to imaginary order, $\rfun{\mathrm{F}_{\nu}}{\sigma}$ and
$\rfun{\mathrm{G}_{\nu}}{\sigma}$ are defined in %\footfullcite{Dunster1990}.
\end{itemize}
\end{frame}

\begin{frame}
{Physical mode functions}%
%{233}
\begin{itemize}
\item Physical mode functions are expected to be regular on the boundary.
\item $\rbr{+,+}$: $\vbr{\rfun{\mathrm{I}_{\ii\nu}}{\sigma}} \to +\infty$
as $\alpha \to +\infty$
\item $\rbr{-,+}$: 
\begin{itemize}
\item
$\forall n \in \BbbN$, $\vbr{\rfun{\mathrm{Y}_{n}}{\sigma}} \to +\infty$ as 
$\alpha \to -\infty$.
\item
$\forall\nu\in\BbbR^+\backslash\BbbN$, choose $\mathrm{J}_{-\nu}$ instead of 
$\mathrm{Y}_{\nu}$, since $\mathrm{J}_{\pm\nu}$ are also linearly independent.
\item
$\forall\nu\in\BbbR^+\backslash\BbbN$, $\vbr{\rfun{\mathrm{J}_{-\nu}}{\sigma}} 
\to +\infty$ as $\alpha \to -\infty$.
\end{itemize}
\item $\rbr{-,-}$: $\vbr{\rfun{\mathrm{K}_{\nu}}{\sigma}} \to +\infty$
as $\alpha \to -\infty$; $\vbr{\rfun{\mathrm{I}_{\nu}}{\sigma}} \to +\infty$
as $\alpha \to +\infty$
\item These are not to be included in the space of physical wave 
functions. $\forall \nu \ge 0$,
\begin{itemize}
\item $\rbr{+,+}$: $\rfun{\mathrm{K}_{\ii\nu}}{\sigma}$ survives
\item $\rbr{+,-}$: $\rfun{\mathrm{F}_{\ii\nu}}{\sigma}$ and
$\rfun{\mathrm{G}_{\ii\nu}}{\sigma}$ survives
\item $\rbr{-,+}$: $\rfun{\mathrm{J}_{\nu}}{\sigma}$ survives
\item $\rbr{-,-}$: drops out
\end{itemize}
\end{itemize}

\end{frame}

\subsection{Semi-classical approximation}

\begin{frame}%
{Matching quantum number with classical first integral}%
{Principle of constructive interference}
\begin{itemize}
\item Write the mode function in the WKB form
\begin{align}
\rfun{\Psi_{k_\beta}}{\beta,\chi} \sim \sqrt{\rfun{\rho}{\beta,\chi}}\,
\cfun{\exp}{\frac{\ii}{\hslash}\rfun{S}{\beta,\chi}}.
\end{align}

\item For $S/\hslash \gg 1$ and $k_\beta \gg 1$\footnote{The common form is
$\hslash\to 0^+$.}, $\rfun{S}{\beta,\chi}$ becomes the Hamilton principle
function in the leading-order approximation.

\item A Hamilton principle function is stationary with respect to variation of
integral constants%\footfullcite{Gerlach1969,Landau_1976}
\begin{align}
\frpa{S}{k_\beta} = 0.
\label{eq:principle-const-interf}
\end{align}

\item Demanding \cref{eq:principle-const-interf} matching the classical
trajectory, $k_\beta$ can be related to $p_\beta$.
\end{itemize}
\end{frame}

\begin{frame}%
{Matching quantum number with classical first integral}%
{$\rbr{+,+}$ as exemplar}
\begin{itemize}
\item Fixing $\nu/\sigma>1$, the asymptotic expansion
reads
\begin{align}
\rfun{\mathrm{K}_{\ii \nu}}{\sigma} \sim
\frac{\sqrt{2\pp}\,
\rfun{\cos}{\sqrt{\nu^2-\sigma^2}-\nu\arccos\frac{\nu}{\sigma}-\frac{\pp}{4}}
}{
\rbr{\nu^2-\sigma^2}^{1/4}\ee^{\pp\nu/2}}
+\rfun{\Omicron}{\frac{1}{\sigma}}.
\end{align}
\begin{itemize}
\item There are two phases with opposite signs.
%$\rfun{\theta_\nu^{\pm}}{\sigma}$.
Assuming $c_i$, $a_j$'s are real and applying \emph{the principle} to 
$\rfun{\Psi_\nu}{\sigma}$, one has$\sigma/\nu = 
\rfun{\sech}{\mscrs_\beta\gamma}$, which matches the trajectory with
$\beta_0 = 0$ if
\begin{align}
\hslash k_\beta \equiv \hslash \sqrt{\frac{3}{2\varkappa}}g\hslash\nu
= p_\beta,
\end{align}

\item Non-vanishing $C$ can be compensated by the phase of $c_i$ and $a_j$'s.
\end{itemize}

\item Fixing $\nu/\sigma<1$, the asymptotic expansion is not oscillatory,
but exponential; it is not within the WKB regime.
\end{itemize}

\begin{itemize}
\item The conclusions also hold for $\rfun{\mathrm{F}_{\ii\nu}}{\sigma}$, 
$\rfun{\mathrm{G}_{\ii\nu}}{\sigma}$ for $\rbr{+,-}$, and
$\rfun{\mathrm{J}_{\nu}}{\sigma}$ for $\rbr{-,+}$.
\end{itemize}
\end{frame}

\subsection{Inner product and wave packet}

\begin{frame}%
{Inner product for wave functions}%
{Schrödinger product}
\begin{itemize}
\item To make sense of amplitude and wave packet, an inner product is necessary

\item In terms of a Klein--Gordon equation, call $\beta$ the ``temporal'' 
variable, and $\chi$ the ``spacial'' variable.

\item A common starting point is the
\emph{Schrödinger product}%\footfullcite[ch.\ 5]{Kiefer2012}
\begin{align}
\rbr{\Psi_1,\Psi_2}_\text{S} \coloneqq
\int \dd \chi\,\rfun{\Psi_1^*}{\beta,\chi} \rfun{\Psi_2}{\beta,\chi};
\end{align}
\item $\rbr{\Psi, \Psi}_\text{S}$ is \textcolor{blue}{positive-definite},
and the integrand $\rfun{\rho_\text{S}}{\beta,\chi}$ is
\textcolor{blue}{non-negative}

%\item In terms of a norm, $\rbr{\Psi, \Psi}_\text{S} \equiv \int \dd \chi\,
%\rfun{\rho_\text{S}}{\beta,\chi},$ in which $\rho_\text{S} \coloneqq 
%\Psi^*\Psi$. Manifestly $\rho_\text{S} \textcolor{blue}{\ge 0}$, and
%$\rbr{\Psi, \Psi}_\text{S}$.

\item The corresponding Schrödinger current is \textcolor{blue}{real} but
\textcolor{orange}{does not satisfy continuity equation}
$\dot{\rho}_\text{S} + \nabla\cdot \vec{j}_\text{S} = 0$, because 
\cref{eq:WDW-10} is KG-like.
\item $\mathrm{K}_{\ii\nu}$%
%\footfullcite{Yakubovich2006,Passian2009,Szmytkowski2010} for $\rbr{+,+}$, 
$\mathrm{F}_{\ii\nu}$ and $\mathrm{G}_{\ii\nu}$ for $\rbr{+,-}$ can be proved 
to be \textcolor{blue}{orthogonal} and \textcolor{blue}{complete}
among themselves, as well as \textcolor{blue}{can be normalised}.
\begin{itemize}
\item $\mathrm{J}_{\ii\nu}$'s for $\rbr{+,-}$ are
\textcolor{orange}{not orthogonal}
\end{itemize}

\end{itemize}
\end{frame}

\begin{frame}%
{Peculiarity of the phantom model $\rbr{-,+}$}%
{Orthogonality for mode functions; Hermiticity for operators}
\begin{itemize}

\item $\rfun{\mathrm{J}_{\nu}}{\sigma}$'s are
\textcolor{orange}{not orthogonal} under the Schrödinger product
\begin{align}
\rbr{\mathrm{J}_{\nu},\mathrm{J}_{\tilde{\nu}}}_\text{S}
\propto \int_{-\infty}^{+\infty} \dd x\,
\rfun{\mathrm{J}_{\nu}^*}{\ee^x}
\rfun{\mathrm{J}_{\tilde{\nu}}}{\ee^x} =
\frac{2\rfun{\sin}{\pp\rbr{\nu-\tilde{\nu}}/2}}{\pp\rbr{\nu^2-\tilde{\nu}^2}},
\end{align}
therefore $\BbbD$ in \cref{eq:separated-10} is
\textcolor{orange}{not Hermitian} (though we do not need it so far)

\item $\widehat{p}_\chi^2$ is \textcolor{orange}{not Hermitian} for
$\cbr{\rfun{\mathrm{J}_{\nu}}{\sigma}}$ under the Schrödinger product
\begin{align}
\int_{-\infty}^{+\infty} \dd x\,
\mathrm{J}_{\nu}^*\rbr{-\partial_x^2\mathrm{J}_{\tilde{\nu}}}
-\int_{-\infty}^{+\infty} \dd x\,
\rbr{-\partial_x^2\mathrm{J}_{\nu}}^*\mathrm{J}_{\tilde{\nu}}
= \frac{2}{\pp}\sin\frac{\pp\rbr{\nu-\tilde{\nu}}}{2}.
\end{align}

\item In order to save Hermiticity for $p_\chi^2$ and $\BbbD$ and 
orthogonality of the modes under Schrödinger product, one can restrict
\begin{align}
\nu = 2n+\nu_0,\qquad \alert{n\in\BbbN},\quad\nu_0\in \left[0,2\right).
\end{align}

\item Using classical singularities as boundary condition, one can fix
$\nu_0 = 1$.
\end{itemize}

\end{frame}

\begin{frame}%
{Discretisation of the phantom model $\rbr{-,+}$}%
%{233}
Levels of the phantom model are \alert{discretised} if one imposes Hermiticity 
of squared momenta under the Schrödinger product.

\begin{itemize}
\item This kind of subtlety on Hermiticity is named as \emph{self-adjoint 
extension}, which arises already for infinite square 
well%\footfullcite{Bonneau2001}.
\item It also applies to $x^{-2}$ potentials%
%\footfullcite{Essin2006,Araujo2004}, which is of cosmological relevance%
%\footfullcite{Bouhmadi-Lopez2009}.
\end{itemize}

\end{frame}


\begin{frame}%
{Further inner products for wave functions}%
{Klein--Gordon and Mostafazadeh product}
\begin{itemize}
\item Since \cref{eq:WDW-10} is KG-like, another popular choice is the KG 
product
\begin{align}
\rbr{\Psi_1,\Psi_2}_\text{KG}^g \coloneqq \ii g \cbr{
\rbr{\Psi_1,\partial_\beta\Psi_2}_\text{S} -
\rbr{\partial_\beta\Psi_1,\Psi_2}_\text{S}},\qquad g > 0.
\end{align}
\begin{itemize}
\item $\rbr{\Psi,\Psi}_\text{KG}^g$ is \textcolor{blue}{real} but
\textcolor{orange}{not positive-definite}, so does the integrand
$\rho_\text{KG}$;
%is \textcolor{blue}{real} but may go \textcolor{orange}{negative},
\item The corresponding $\vec{J}_\text{KG}$ is
\textcolor{blue}{conserved} $\dot{\rho}_\text{KG} +
\nabla\cdot \vec{J}_\text{KG} = 0$ and \textcolor{blue}{real}.
\end{itemize}

\item Mostafazadeh%\footfullcite{Mostafazadeh2002} found
% a class of
a product \emph{for Hermitian $\BbbD$ with positive spectrum}:
\begin{align}
% \rbr{\Psi_1,\Psi_2}_\text{M}^{\kappa,a} &\coloneqq
% \rbr{\Psi_1,\Psi_2}_\text{M}^{\kappa} +
% \kappa \rbr{\Psi_1,\Psi_2}_\text{KG}^a,\\
\rbr{\Psi_1,\Psi_2}_\text{M}^{\kappa} &\coloneqq\kappa\cbr{
\rbr{\Psi_1,\BbbD^{+1/2}\Psi_2}_\text{S}
+\rbr{\partial_\beta\Psi_1,
\BbbD^{-1/2}\partial_\beta\Psi_2}_\text{S}}, \qquad \kappa > 0.
\end{align}
% and $-1 < a < 1$.
\begin{itemize}

\item $\rbr{\Psi,\Psi}_\text{M}^{\kappa}$ is
\textcolor{blue}{positive-definite}, but the integrand $\rho_\text{M}^\kappa$
is \textcolor{orange}{complex}
\item The corresponding $\vec{J}_\text{M}^\kappa$ is
\textcolor{blue}{conserved} $\dot{\rho}_\text{M}^\kappa + 
\nabla\cdot \vec{J}_\text{M}^\kappa = 0$ but also \textcolor{orange}{complex}.
\end{itemize}
\end{itemize}
\end{frame}

\begin{frame}%
{Mostafazadeh inner product and the corresponding density}%
%{233}
\begin{itemize}
\item Real power of $\BbbD$ is defined by spectral decomposition
$\BbbD^{\gamma} \coloneqq \sum_\nu \nu^{2\gamma}\mbfP_\nu$,
$\mbfP_\nu\Psi \coloneqq \Psi_\nu\rbr{\Psi_\nu,\Psi}_\text{S}$.
\end{itemize}
\begin{itemize}
\item It can be shown%\footfullcite{Mostafazadeh2006} that \emph{the density}
\begin{align}
\varrho_\text{M}^\kappa \coloneqq\kappa\cbr{
\vbr{\BbbD^{+1/4}\Psi}^2+\vbr{\BbbD^{-1/4}\partial_\beta{\Psi}}^2}
\end{align}
\begin{itemize}
\item is \textcolor{blue}{equivalent to $\rho_\text{M}^\kappa$} up to a
boundary term
\begin{align}
\int\dd\chi\,\varrho_\text{M}^\kappa = 
\int\dd\chi\,\rho_\text{M}^\kappa \equiv
\rbr{\Psi_1,\Psi_2}_\text{M}^{\kappa};
\end{align}
\item is \textcolor{blue}{non-negative}.
%\item may be understood as a probability density;
\item The corresponding current $\vec{\mscrJ}_\text{M}^\kappa$
is \textcolor{blue}{real} but
\textcolor{orange}{not conserved}%\footfullcite{Rosenstein1985}.
\end{itemize}

\end{itemize}
\end{frame}

\begin{frame}%
{Wave packets of Gaussian amplitude for continuous spectrum}%
{Quintessence models}
\begin{itemize}
\item It is difficult to find an amplitude such that the wave packet is
Gaussian
\item Instead, one can choose a Gaussian amplitude
\begin{align}
\rfun{A}{\nu; \overline{\nu},\sigma} \coloneqq
\rbr{\frac{1}{\sqrt{2\pp}\,\sigma}\rfun{\exp}{
-\frac{\rbr{\nu-\overline{\nu}}^2}{2\sigma^2}}}^{1/2}
\end{align}

\item In %\footfullcite{Dabrowski2006},
$\rfun{A}{\nu;\overline{\nu},\sigma/\sqrt{2}}$ was chosen.

\end{itemize}
\end{frame}

\begin{frame}%
{Wave packets of Gaussian amplitude for quintessence model $\rbr{+,+}$}%
{$\mathrm{K}_{\ii\nu}$, with $\lambda = \varkappa^{1/2}/2$,
$V = -\varkappa^{-2}$, $\overline{k}_\beta = -2$ and $\sigma_\beta = 5/4$}
\begin{columns}
\begin{column}{.49\textwidth}
\begin{block}{Schrödinger}

\end{block}
\end{column}
\begin{column}{.49\textwidth}
\begin{block}{Mostafazadeh}
%\only<1>{
%\includegraphics[width=\textwidth]{../plots.paper.nb/mosta_cosh_2d.pdf}}
%\only<2>{
%\includegraphics[width=\textwidth]{../plots.paper.nb/mosta_cosh_3d.pdf}}
\end{block}
\end{column}
\end{columns}
\end{frame}

\begin{frame}%
{Wave packets of Gaussian amplitude for quintessence model $\rbr{+,-}$}%
{$\mathrm{F}_{\ii\nu}$, with $\lambda = 4\varkappa^{1/2}/5$,
$V = +\varkappa^{-2}$, $\overline{k}_\beta = -7/2$ and $\sigma_\beta = 7/5$}
\begin{columns}
\begin{column}{.49\textwidth}
\begin{block}{Schrödinger}

\end{block}
\end{column}
\begin{column}{.49\textwidth}
\begin{block}{Mostafazadeh}

\end{block}
\end{column}
\end{columns}
\end{frame}

\begin{frame}%
{Wave packets of Gaussian amplitude for quintessence model $\rbr{+,-}$}%
{$\mathrm{G}_{\ii\nu}$, with $\lambda = 4\varkappa^{1/2}/5$,
$V = +\varkappa^{-2}$, $\overline{k}_\beta = -7/2$ and $\sigma_\beta = 7/5$}
\begin{columns}
\begin{column}{.49\textwidth}
\begin{block}{Schrödinger}

\end{block}
\end{column}
\begin{column}{.49\textwidth}
\begin{block}{Mostafazadeh}

\end{block}
\end{column}
\end{columns}
\end{frame}

\begin{frame}%
{Wave packets with Poissonian amplitude for discrete spectrum}%
{Discrete phantom model $\rbr{-,+}$}
\begin{itemize}
\item Gaussian distribution works for continuous variable

\item For discrete spectrum, one can choose a Poissonian amplitude
\begin{align}
\rfun{A_n}{\overline{n}} \coloneqq
\rbr{\ee^{-\overline{n}}\frac{\overline{n}^n}{n!}}^{1/2}
\end{align}
\item In %\footfullcite{Kiefer1990},
$\rfun{A_n}{\overline{n}/\sqrt{2}}$ was chosen.

\end{itemize}
\end{frame}

\begin{frame}%
{Wave packets of Poissonian amplitude for phantom model}%
{$\mathrm{J}_{2n+1}$, with $\lambda = 2\varkappa^{1/2}$,
$V = +\varkappa^{-2}$ and $\overline{k}_\beta = 8$}
\begin{columns}
\begin{column}{.49\textwidth}
\begin{block}{Schrödinger}
%\only<1>{
%\includegraphics[width=\textwidth]{../plots.paper.nb/naive_cos_1_2d_m.pdf}}
%\only<2>{
%\includegraphics[width=\textwidth]{../plots.paper.nb/naive_cos_1_3d_m.pdf}}
\end{block}
\end{column}
\begin{column}{.49\textwidth}
\begin{block}{Mostafazadeh}
%\only<1>{
%\includegraphics[width=\textwidth]{../plots.paper.nb/mosta_cos_1_2d_m.pdf}}
%\only<2>{
%\includegraphics[width=\textwidth]{../plots.paper.nb/mosta_cos_1_3d_m.pdf}}
\end{block}
\end{column}
\end{columns}
\end{frame}

\begin{frame}%
{Wave packets of Gaussian amplitude for phantom model}%
{$\mathrm{J}_{\nu}$, with $\lambda = 2\varkappa^{1/2}$,
$V = +\varkappa^{-2}$, $\overline{k}_\beta = 8$ and $\sigma_\beta = 11/2$}
\begin{columns}
\begin{column}{.49\textwidth}
\begin{block}{Schrödinger}
%\only<1>{
%\includegraphics[width=\textwidth]{../plots.paper.nb/naive_cos_gauss_2d.pdf}}
%\only<2>{
%\includegraphics[width=\textwidth]{../plots.paper.nb/naive_cos_gauss_3d.pdf}}
\end{block}
\end{column}
\begin{column}{.49\textwidth}
\begin{block}{Mostafazadeh}
%\only<1>{
%\includegraphics[width=\textwidth]{../plots.paper.nb/mosta_cos_gauss_2d.pdf}}
%\only<2>{
%\includegraphics[width=\textwidth]{../plots.paper.nb/mosta_cos_gauss_3d.pdf}}
\end{block}
\end{column}
\end{columns}
\end{frame}

\section{Conclusions}

\begin{frame}%
{Highlights}%
%{233}
\begin{itemize}
\item An \alert{integral of motion} was found for Liouville cosmological
models.
\item \alert{Implicit trajectories} in minisuperspace were obtained.
\item The levels of phantom Liouville model were found to be \alert{discrete}
due to the Hermiticity requirement of observables.
\item The semi-classical quantum mode functions and wave packets was compared
directly with the classical trajectory.
\end{itemize}
\end{frame}

\begin{frame}%
{Issues}%
%{233}
\begin{itemize}
%\item Dimension of wave packets sloppy
\item In $\rbr{+,-}$ and $\rbr{-,+}$, wave packets contain multiple branches;
however, the classical universe runs only on one trajectory.
\item Quantum-corrected $\overline{k}_\beta$ is to be understood.
\item Wave packets with Gaussian profile are to be constructed, instead of
inserting Gaussian / Poissonian amplitude artificially.
\item A normalising $\kappa$ for $\rbr{\cdot,\cdot}_\text{M}^\kappa$ is to be 
evaluated, otherwise a quantitative comparison of
$\rbr{\cdot,\cdot}_\text{S}$ and $\rbr{\cdot,\cdot}_\text{M}^\kappa$ is not
possible.
\end{itemize}
\end{frame}

\begin{frame}%
{Outlook}%
%{233}
\begin{itemize}
\item Beyond isotropy: generalise to Bianchi models
\begin{itemize}
\item Bianchi Type-I: a natural extension, \alert{under investigation}
\end{itemize}

\item Beyond homogeneity: cosmological perturbation,
\alert{under investigation}

\item Beyond single field
\begin{itemize}
\item Two exponential potentials: $\vbr{V_1} = \vbr{V_2}$ and special 
$\lambda_i$
\item Multiple Liouville fields: mixing kinetic
terms needed%\footfullcite{Andrianov2015}
\end{itemize}

\item Beyond classic matter
\begin{itemize}
\item $PT$-symmetric Liouville 
field%\footfullcite{ANDRIANOV2010,Andrianov2016}: 
may cross the phantom divide $w = -1$.
\end{itemize}

\end{itemize}
\end{frame}




%\begin{frame}[fragile]
%  \frametitle{Laden des Themes}
%  \begin{block}{Das Theme kann mit den folgenden Optionen geladen werden}
%    \begin{small}
%\begin{verbatim}
%\usetheme[%
%% uk,      %% Farben aller Fakultaeten
%wiso,     %% Wiso-Fakultaet
%% jura,    %% Rechtswissenschaftliche Fakultaet
%% medizin, %% Medizinische Fakultaet
%% philo,   %% Philosophische Fakultaet
%% matnat,  %% Mathematisch-Naturwissenschaftliche Fakultaet
%% human,   %% Humanwissenschaftliche Fakultaet
%% verw,    %% Universitaetsverwaltung
%%]{UzK}
%\end{verbatim}
%    \end{small}
%
%  \end{block}
%\end{frame}

% \begin{frame}
%   \frametitle{Die Fußzeile}
% 
%   \begin{itemize}
%   \item Es stehen verschiedene Fußzeilen zur Auswahl, die als Option
%     beim Laden des \emph{themes} übergeben werden:
%     \begin{itemize}
%     \item Balken mit allen Fakultätsfarben (Option \texttt{uk})
%     \item Balken in jeweils einer Fakultätsfarbe (Optionen \texttt{wiso, jura,
%         medizin, philo, matnat, human, verw})\footnote{Es werden die 
% offiziellen
%         RGB-Werte aus dem 2-D Handbuch Corporate Design verwendet.}
%     \end{itemize}
%   \item "`Universität zu Köln"' sowie der Name der Fakultät sind im
%     Theme definiert, das Institut oder Seminar kann mit dem Befehl
%     \texttt{\textbackslash institute\{\}} festgelegt werden
%   \item Die Optionen sind im Quellcode dieser Präsentation dokumentiert
%   \end{itemize}

% \end{frame}

% \begin{frame}
%   \frametitle{Englische Präsentationen}
%   \begin{itemize}
%   \item Der Universitäts- sowie die Fakultätsnamen werden
%     standardmäßig auf Deutsch angezeigt.
%   \item Übergeben Sie dem Paket \texttt{babel} die Option
%     \texttt{english}, so werden diese Namen entsprechen angepasst.
%   \item Die Übersetzungen können in der Theme-Datei
%     \texttt{beamerthemeUzK.sty} geändert werden
%   \end{itemize}
% 
% \end{frame}

%\begin{frame}
%  \frametitle{\texttt{block}-Umgebungen}
%  \begin{block}{Standard (\texttt{block})}
%    Verwendet die Farbe "`Blaugrau Mittel"' als Blocktitel-Hintergrund
%  \end{block}
%
%  \begin{exampleblock}{\texttt{exampleblock}}
%    Bei Verwendung der Fußzeile mit allen Fakultätsfarben
%    Titelhintergrund in Wiso-Grün, sonst in der jeweiligen
%    Fakultätsfarbe
%  \end{exampleblock}
%
%  \begin{alertblock}{\texttt{alertblock}}
%    Verwendet das Rot der Folientitel
%  \end{alertblock}
%
%\end{frame}


% \begin{frame}
%   \frametitle{Installation}
%   \begin{itemize}
%   \item Das Theme besteht aus den Dateien
%     \texttt{beamerthemeUzK.sty} und \texttt{beamercolorthemeUzK.sty}
%     sowie den Grafikdateien \texttt{logo.pdf} und
%     \texttt{logo-small.pdf}.
%   \item Das Theme kann auf zwei Arten verwendet werden:
%     \begin{enumerate}
%     \item Die vier Dateien werden in den selben Ordner wie die zu
%       erstellende Präsentation gelegt
%     \item Die vier Dateien werden im lokalen \emph{texmf}-Baum abgelegt
%     \end{enumerate}
%   \item Die zweite Variante ist der ersten vorzuziehen, da das Theme
%     so an einem zentralen Ort vorliegt
%   \end{itemize}
% \end{frame}


% \begin{frame}
%   \frametitle{ToDo}
% 
%   \begin{block}{Was noch zu tun ist\ldots}
%     \begin{itemize}
%     \item Erstellen einer eigenen Titelseite
%     \item \ldots
%     \end{itemize}
%   \end{block}
% 
% \end{frame}

\end{document}
