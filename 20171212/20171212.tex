%% UzK - A BEAMER THEME FOR THE UNIVERSITY OF COLOGNE
%% http://solstice.github.com/uzk-theme/

\documentclass[9pt]{beamer}

\usepackage{fontspec}

% Multilingual support
\usepackage{polyglossia}

% more symbols
\usepackage{textcomp}

% fontspec, realscripts, metalogo fot XeLaTeX
\usepackage{xltxtra}
	% Unicode fonts
	\setmainfont{CMU Serif}
	\setsansfont{CMU Sans Serif}
	\setmonofont{CMU Typewriter Text}
\usepackage{amsfonts}
\usepackage{unicode-math}
	\setmathfont{Latin Modern Math} % default
	\setmathfont{Latin Modern Math}[range={
		"1D608-"1D63B},	% Italic sans serif, Latin, uppercase
		sans-style=italic]
	\setmathfont{Latin Modern Math}[range={
		"1D6E2-"1D6FA,	% Italic Greek, uppercase
		"1D6FC-"1D71B},	% Italic Greek, lowercase
		math-style=ISO]
	\setmathfont{Latin Modern Math}[range={
		"00391-"003A9,	% Upright Greek, uppercase
		"003B1-"003F5,	% Upright Greek, lowercase
		"1D6A8-"1D6E1},	% Bold Greek
		sans-style=upright]
	\setmathfont{Asana Math}[range={
		\mathbin}] %\mathord
	\setmathfont{STIX Math}[range={
		"02609}] % ☉
	\setmathfont{XITS Math}[range={
		"1D4B6-"1D4CF}] % Script, Latin, lowercase
	%\setmathfont{⟨font name⟩}[range=⟨unicode range⟩,⟨font features⟩]

\usepackage{xeCJK}
\setmainlanguage{english}
\setotherlanguage{ngerman}

\usefonttheme{professionalfonts}
%  don't change fonts inside beamer

% AMS--related
\usepackage{amsmath,amssymb}

% ':=' as \coloneqq
\usepackage{mathtools}
\usepackage{siunitx}
\usepackage{cleveref}

% some unicode characters
% ≙ for equal with hat


% Mathematical constants
\newcommand{\ii}{{\Bbbi}}
\newcommand{\ee}{{\Bbbe}}
\newcommand{\pp}{{\Bbbpi}}

% Math operators
\DeclareMathOperator{\arcsinh}{arcsinh}
\DeclareMathOperator{\arccosh}{arccosh}
\DeclareMathOperator{\arctanh}{arctanh}
\DeclareMathOperator{\arccoth}{arccoth}
\DeclareMathOperator{\arcctgh}{arcctgh}
\DeclareMathOperator{\arcsech}{arcsech}
\DeclareMathOperator{\arccsch}{arccsch}

\DeclareMathOperator{\BesselJ}{J}
\DeclareMathOperator{\BesselY}{Y}
\DeclareMathOperator{\BesselF}{F}
\DeclareMathOperator{\BesselG}{G}
\DeclareMathOperator{\BesselI}{I}
\DeclareMathOperator{\BesselK}{K}
\DeclareMathOperator{\BesselL}{L}

\DeclareMathOperator{\sgn}{sgn}
\DeclareMathOperator{\grad}{grad}
\DeclareMathOperator{\curl}{curl}
\DeclareMathOperator{\rot}{rot}
\DeclareMathOperator{\opdiv}{div}
\DeclareMathOperator{\opdeg}{deg}

\DeclareMathOperator{\sech}{sech}
\DeclareMathOperator{\csch}{csch}

\DeclareMathOperator{\diag}{diag}
\DeclareMathOperator{\tr}{tr}
\DeclareMathOperator{\rank}{rank}

\DeclareMathOperator{\ad}{ad}

\DeclareMathOperator{\expi}{expi}

% Differentials
\newcommand{\DD}{\BbbD}
\newcommand{\dd}{\Bbbd}
\newcommand{\dva}{\mupdelta} % no better way?!
\newcommand{\Dva}{\mupDelta}

% Equal marks
\newcommand{\eeq}{{\overset{!}{=}}}
\newcommand{\lls}{{\overset{!}{<}}}
\newcommand{\ggt}{{\overset{!}{>}}}
\newcommand{\lle}{{\overset{!}{\le}}}
\newcommand{\gge}{{\overset{!}{\ge}}}

% Group and Algebras
\newcommand{\SO}{\msansS\msansO}
\newcommand{\SU}{\msansS\msansU}
\newcommand{\so}{\mfraks\mfrako}
\newcommand{\su}{\mfraks\mfraku}
% Bracket-like
\newcommand{\rbr}[1]{{\left(#1\right)}}
\newcommand{\sbr}[1]{{\left[#1\right]}}
\newcommand{\cbr}[1]{{\left\{#1\right\}}}
\newcommand{\abr}[1]{{\left<#1\right>}}
\newcommand{\vbr}[1]{{\left|#1\right|}}
\newcommand{\dvbr}[1]{{\left\|#1\right\|}}
\newcommand{\fat}[2]{{\left.#1\right|_{#2}}}
% Functions; note the space between the name and the bracket!
\newcommand{\rfun}[2]{{#1}\mathopen{}\left(#2\right)\mathclose{}}
\newcommand{\sfun}[2]{{#1}\mathopen{}\left[#2\right]\mathclose{}}
\newcommand{\cfun}[2]{{#1}\mathopen{}\left\{#2\right\}\mathclose{}}
\newcommand{\afun}[2]{{#1}\mathopen{}\left<#2\right>\mathclose{}}
\newcommand{\vfun}[2]{{#1}\mathopen{}\left|#2\right|\mathclose{}}
% Fraction-like
\newcommand{\frde}[2]{{\frac{\dd{#1}}{\dd{#2}}}}
\newcommand{\frDe}[2]{{\frac{\DD{#1}}{\DD{#2}}}}
\newcommand{\frpa}[2]{{\frac{\partial{#1}}{\partial{#2}}}}
\newcommand{\frdva}[2]{{\frac{\dva{#1}}{\dva{#2}}}}

% overline-like marks
\newcommand{\ol}[1]{{\overline{{#1}}}}
\newcommand{\ul}[1]{{\underline{{#1}}}}
\newcommand{\tld}[1]{{\widetilde{{#1}}}}
\newcommand{\ora}[1]{{\overrightarrow{#1}}}
\newcommand{\ola}[1]{{\overleftarrow{#1}}}
\newcommand{\td}[1]{{\widetilde{#1}}}
\newcommand{\what}[1]{{\widehat{#1}}}
%\newcommand{\prm}{{\symbol{"2032}}}

% Physical constants and parameters
\newcommand{\lc}{\mitsansc} % speed of light in vacuum
\newcommand{\bk}{\mitsansk} % Boltzmann's constant
\newcommand{\phs}{\hslash} % reduced Planck constant
\newcommand{\ph}{\Planckconst} % Planck constant

\newcommand{\nG}{\mitsansG} % Newton's constant
\newcommand{\aN}{\mitsansN} % Avogadro number
\newcommand{\ec}{\mitsanse} % unit electric charge

\newcommand{\gR}{\mitsansR} % gas constant

\newcommand{\plm}{m_\text{P}} % Planck mass
\newcommand{\pll}{l_\text{P}} % Planck length
\newcommand{\plt}{t_\text{P}} % Planck time

\newcommand{\hH}{\mitsansH} % Hubble parameter H
\newcommand{\hh}{\mitsansh} % Hubble parameter h
\newcommand{\dq}{\mitsansq} % Deceleration parameter q

\newcommand{\apE}{\alpha_\text{E}} % EM fine struct const
\newcommand{\apG}{\alpha_\text{G}} % Grav fine struct const

% Common symbols
\newcommand{\Ld}{\mscrL} % Lagrangian density
\newcommand{\fp}{p_\text{F}} % Fermi momentum
\newcommand{\fE}{\mscrE_\text{F}} % Fermi energy

% Others
\newcommand{\rSch}{R_\text{S}} % Schwarzschild radius

\newcommand{\fHor}{{\mscrh^+}} % future horizon
\newcommand{\pHor}{{\mscrh^-}} % past horizon

% Chemical elements
\usepackage[version=4]{mhchem}


% siunitx
% Astronomy
\DeclareSIUnit\parsec{pc}
\DeclareSIUnit\lightyear{ly}

\usepackage[%style=numeric,
			isbn = false, doi = false,
			backend=biber]{biblatex}
\addbibresource{20171212.bib}
\AtEveryCitekey{\clearfield{title}\clearfield{issn}}

%% Falls Anzeige der \sections, \subsections etc. gewuenscht, kann zB.
%% das infolines theme geladen werden. Wichtig ist jedoch, dass andere
%% Themes _vor_ dem UzK-Theme geladen werden.
%\useoutertheme{infolines}

%% Falls keine der Optionen zur Bestimmung der Fusszeile uebergeben werden    %%
%% werden alle Fakultaetsfarben verwendet. ---------------------------------- %%
\usetheme[%
%wiso,        %% Wiso-Fakultaet
%jura,        %% Rechtswissenschaftliche Fakultaet
%medizin,     %% Medizinische Fakultaet
%philo,       %% Philosophische Fakultaet
%matnat,      %% Mathematisch-Naturwissenschaftliche Fakultaet
%human,       %% Humanwissenschaftliche Fakultaet
%verw,        %% Universitaetsverwaltung
%nav,         %% Schaltet die Navigationssymbole ein
latexfonts,  %% Verwendet die latex-beamer-Standardschrift
%colorful,    %% Farbige Balken im infolines-Theme
%squares,     %% Aufzaehlungspunkte rechteckig
%nologo,      %% Kein Logo im Seitenhintergrund
]{UzK}

\title{Integrable Cosmological Models with Liouville Fields}

\author[Andrianov \and Lan \and Novikov \and \underline{Wang}]{
	Alexander A. Andrianov\inst{1,4} %\thanks{a.andrianov@spbu.ru}
	\and
	Chen Lan\inst{2} %\thanks{stlanchen@yandex.ru}
	\and
	Oleg O. Novikov\inst{1} %\thanks{o.novikov@spbu.ru}
	\and 
	\underline{Yi-Fan Wang}\inst{3}} %\thanks{yfwang@thp.uni-koeln.de}
%{David Kusterer\thanks{ 
%\href{mailto:kusterer@uni-koeln.de}{kusterer@uni-koeln.de} }%
%  \and%
%  Bernd 
%Weiß\thanks{\href{mailto:bernd.weiss@wiso.uni-koeln.de}{
%bernd.weiss@wiso.uni-koeln.de}}}

%\institute[Forschungsinstitut für Soziologie]{%
%Forschungsinstitut für Soziologie \\
%Greinstraße 2\\
%50939 Köln}

\institute[SPBU \and ELI-ALPS \and UzK \and UB]{
\inst{1} Saint-Petersburg State University, St.\ Petersburg 198504, Russia \and
\inst{2} ELI-ALPS, ELI-Hu NKft, Dugonics t\'er 13, Szeged 6720, Hungary \and
\inst{3} Institut f\"ur Theoretische Physik, Universit\"at zu K\"oln,
Z\"ulpicher Stra\ss e 77, 50937 K\"oln, Germany \and
\inst{4}
Institut de Ci\`encies del Cosmos (ICCUB), Universitat de Barcelona, Spain}


\begin{document}

\begin{frame}%[titlepage]
  \titlepage
\end{frame}

\begin{frame}{Outline}
  \tableofcontents
  % You might wish to add the option [pausesections]
\end{frame}


\section{Introduction}

\begin{frame}%
{Introduction}%
{Introduction}
\begin{itemize}
	\item
\end{itemize}
\end{frame}

\section{Classical model and the implicitised trajectories}

\begin{frame}%
{The Friedmann-Lemaître model}%
{123}
\begin{itemize}
\item Flat Robertson--Walker metric
$\dd s^2 = -\rfun{N^2}{t}\,\dd t^2
+\varkappa^{-1}\ee^{2\rfun{\alpha}{t}}\,\dd\Omega_3^2$,
where $\varkappa \coloneqq 8\pp\nG$, $\dd\Omega_3^2$ dimensionless spacial 
metric.

\item Homogeneous real Klein--Gordon field with potential
$V\ee^{\lambda\phi}$ (Liouville), where $\lambda, V\in \BbbR$,
and kinetic term with sign $\mscrl = \pm 1$ (quintessence
/ phantom model).

\item Total action $\mscrS \coloneqq S_{\text{EH}} + S_\text{GHY} + S_\text{L}
= \int\dd\Omega_3^2\int\dd t\,L$, in which the effective Lagrangian reads
\begin{align}
%S_{\text{Einstein--Hilbert}} + S_\text{Gibbons--Hawking--York} + 
%S_\text{Liouville}
L&\coloneqq \varkappa^{3/2}N\ee^{3\alpha}
\rbr{-\frac{3}{\varkappa}\frac{\dot{\alpha}^2}{N^2}
+\mscrl\frac{\dot{\phi}^2}{2N^2}-V\ee^{\lambda\phi}},
\label{eq:eff-lag-10}
\end{align}
in which dot means $\dd/\dd t$ and $\mscrl = \pm 1$.
\end{itemize}
\end{frame}

\begin{frame}%
{Decoupling the variables}%
{Via \emph{rescaled} special orthogonal transformation}
\begin{itemize}
\item Setting $\overline{N} \coloneqq N\ee^{-3\alpha}$, \cref{eq:eff-lag-10}
%the effective Lagrangian
transforms to
\begin{align}
L = \varkappa^{3/2}\overline{N}
\rbr{-\frac{3}{\varkappa}\frac{\dot{\alpha}^2}{\overline{N}^2}
+\mscrl\frac{\dot{\phi}^2}{2\overline{N}^2} - V\ee^{\lambda\phi+6\alpha}}
\end{align}

\item Defining $\Delta \coloneqq \lambda^2 - 6\mscrl\varkappa$,
$\mscrs \coloneqq \sgn \Delta$ and
$g \coloneqq \mscrs \sqrt{\vbr{\Delta}} \equiv \mscrs\sqrt{\mscrs\Delta}$,
the \emph{rescaled} special orthogonal transformation
\begin{align}
\begin{pmatrix}
\alpha \\ \phi
\end{pmatrix} = \frac{\mscrs}{g}
\begin{pmatrix}
\lambda & -\mscrl\kappa \\
-6 & \lambda
\end{pmatrix}
\begin{pmatrix}
\mscrs_\beta \beta \\ \mscrs_\chi \chi
\end{pmatrix}\qquad\text{where } \mscrs_\beta, \mscrs_\chi = \pm 1
\end{align}
gives the decoupled Lagrangian
\begin{align}
L = \varkappa^{3/2}\overline{N}
\rbr{-\mscrs\frac{3}{\varkappa}\frac{\dot{\beta}^2}{\overline{N}^2}
+\mscrl\mscrs\frac{\dot{\chi}^2}{2\overline{N}^2}
-V\ee^{\mscrs_\chi g\chi}}.
\label{eq:lagrangian-decoupled}
\end{align}

\item
The Euler--Lagrange equations w.r.t.\ $\overline{N}$, $\beta$ and
$\chi$ will be called the trsfed.\ 1st, 2nd Friedmann eqs.\ and the
Klein--Gordon eq., respectively.
\end{itemize}
\end{frame}

\begin{frame}%
{Implicitised integration}%
{$p_\beta \neq 0$}
\begin{itemize}
\item Since $\beta$ is cyclic in \cref{eq:lagrangian-decoupled},
the trsfed.\ 2nd Friedmann eq.\ can be integrated\footfullcite{Lan2016}
\begin{align}
\text{const.} \equiv p_\beta \coloneqq \frpa{L}{\dot{\beta}} &=
-6\mscrs\varkappa^{1/2}\frac{\dot{\beta}}{\overline{N}}
\equiv
-6\mscrs\mscrs_\beta\frac{\varkappa^{1/2}}{g}
\frac{\lambda\dot{\alpha}+\mscrl\varkappa\dot{\phi}}{\overline{N}}.
\end{align}

\item For $p_\beta \neq 0$, fixing the \alert{implicitising gauge}
$\overline{N} = -6\mscrs\sqrt{\varkappa}\dot{\beta}/p_\beta$, the trsfed.\
1st Friedmann equation can be integrated
\begin{align}
\ee^{\mscrs_\chi g\chi} &=
\frac{p_\beta^2}{12\varkappa^2\vbr{V}}
\rfun{S^2}{\mscrs_\beta \sqrt{\frac{3}{2\varkappa}}\,g \beta},
%\ee^{\mscrs_\chi g\chi} &= p_\beta^2/12\varkappa^2\vbr{V}\cdot
%\rfun{f^2}{s_\beta \sqrt{3/2\varkappa}\,g \beta}
%\quad\text{or}\\
%\ee^{6\alpha+\lambda\phi} &= p_\beta^2/12\varkappa^2\vbr{V}\cdot
%\rfun{f^2}{\sqrt{3/2\varkappa}\,\rbr{\lambda\alpha+\mscrl\varkappa\phi}},
\end{align}
in which $\mscrv \coloneqq \sgn V$, and
\begin{align}
\rfun{S}{\gamma} \coloneqq
\begin{cases}
\rfun{\sech}{\gamma+C_{++}} & \rbr{\mscrl, \mscrs\mscrv} = \rbr{+,+}, \\
\rfun{\csch}{\gamma+C_{+-}} & \rbr{\mscrl, \mscrs\mscrv} = \rbr{+,-}, \\
\rfun{\sec}{\gamma+C_{-+}} & \rbr{\mscrl, \mscrs\mscrv} = \rbr{-,+}, \\
\ii\rfun{\csc}{\gamma+C_{--}} & \rbr{\mscrl, \mscrs\mscrv} = \rbr{-,-}. \\
\end{cases}
\end{align}
\end{itemize}
\end{frame}

\begin{frame}%
{Integration}%
{Discussions}
\begin{itemize}
\item The integrals are consistent with the trsfed.\ Klein--Gordon equation.
\item The integral for $\rbr{+,+}$
\begin{itemize}
	\item has two asymptotes
\end{itemize}
\item The implicitised integral for $\rbr{+,-}$ 
\begin{itemize}
	\item contains two distinct solutions
	\item has three asymptotes
\end{itemize}
\item The implicitised integral for $\rbr{-,+}$
\begin{itemize}
	\item contains infinite distinct solutions
	\item has infinite asymptotes, which are pairwise parallel
\end{itemize}
\item The integral for $\rbr{-,-}$
\begin{itemize}
	\item is not real
\end{itemize}
\item The implicitised integral enables one to compare trajectories with wave 
functions, see below.
\end{itemize}
\end{frame}

\begin{frame}%
{Implicitised integration}%
{$p_\beta = 0$}
\begin{itemize}
\item For $p_\beta = 0$, one has $\beta \equiv \beta_0$ or $\phi-\phi_0 =
-\mscrl\lambda\alpha/\kappa$, which is the familiar power-law special
solution\footfullcite{Dabrowski2006}.

\item Further integrating the first Friedmann equation demands $\rbr{+,-}$
or $\rbr{-,+}$ to guarantee $\overline{N} > 0$, and the result is
automatically consistent with the trsfed.\ Klein--Gordon equation.

\item Fixing $\overline{N} = \rbr{2\varkappa^2\vbr{V}}^{-1/2}$ yields
\begin{align}
\ee^{g\mscrs_\chi\chi} = \rbr{\frac{2\kappa}{g\rbr{t-t_0}}}^2.
\end{align}
\end{itemize}
\end{frame}



\section{Dirac quantisation and the wave functions}

\begin{frame}%
{Introduction}%
{Introduction}
\begin{itemize}
	\item
\end{itemize}
\end{frame}

\begin{frame}%
{Dirac quantisation}%
{233}
\begin{itemize}
\item
The primary Hamiltonian and the Hamiltonian constraint reads
\begin{align}
H^\text{p} &= \overline{N}H_\perp + p_{\overline{N}} v^{\overline{N}},
\\
H_\perp &= -\mscrs\frac{p_\beta^2}{12\varkappa^{1/2}}
+\mscrl\mscrs\frac{p_\chi^2}{2\varkappa^{3/2}}
+\varkappa^{3/2}V\ee^{g\mscrs_\chi\chi}.
\end{align}
\item
Applying the Dirac quantisation rules with the Laplace--Beltrami opeator, one 
gets the mss.\ Wheeler--DeWitt eq.\ with $\rbr{\beta,\chi}$
%$\widehat{H}_\perp \rfun{\Psi}{\beta,\chi} = 0$,
\begin{align}
% -\mscrs \frac{12\varkappa^{1/2}}{\hslash^2}\widehat{H}_\perp 
% \Psi \eqqcolon -\partial_\beta^2\Psi - \BbbD\Psi,\qquad
% \BbbD = -\mscrl \frac{6}{\varkappa}\partial_\chi^2 +
% \mscrs \frac{12\varkappa^2 V \ee^{g\mscrs_\chi\chi}}{\hslash^2}.
0 &= \widehat{H}_\perp \rfun{\Psi}{\beta,\chi} \coloneqq
\rbr{\mscrs\frac{\hslash^2}{12\varkappa^{1/2}}\partial_{\beta}^2
-\mscrl\mscrs\frac{\hslash^2}{2\varkappa^{3/2}}\partial_{\chi}^2
+\varkappa^{3/2}V\ee^{g\mscrs_\chi\chi}}\Psi.
\label{eq:WDW-10}
\end{align}
\item
\Cref{eq:WDW-10} is KG-like, hyperbolic for $\mscrl = +1$ and 
\alert{elliptic} for $\mscrl = -1$.
\end{itemize}
\end{frame}

\begin{frame}%
{Separation of the variables and mode functions}%
{233}
\begin{itemize}
\item
Inserting %the separating Ansatz
$\rfun{\Psi}{\beta,\chi} = \ee^{-\ii k_\beta\mscrs_\beta\beta} 
\rfun{\psi}{\chi}$, the remaining
eq. $\BbbD\psi=\nu^2\psi$ turns out to be Besselian, and the mode functions are
\begin{align}
\rfun{\Psi_\nu}{\beta, \chi} \coloneqq
\ee^{-\ii \nu \gamma} \rfun{\psi_\nu}{\chi} \coloneqq
\ee^{-\ii \nu \gamma} \rbr{C_1\rfun{\mathrm{B}_{\nu}^{(1)}}{\sigma}
+C_2\rfun{\mathrm{B}_{\nu}^{(2)}}{\sigma}},
\end{align}
in which
\begin{align}
\nu \coloneqq \sqrt{\frac{2\varkappa}{3}}\frac{k_\beta}{g},\quad
\gamma \coloneqq \sqrt{\frac{3}{2\varkappa}}g\beta,\quad
\sigma^2 \coloneqq 
\frac{8\varkappa^3\vbr{V}\ee^{g\mscrs_\chi\chi}}{\hslash^2 g^2},\\
\label{eq:trsf-quantum-sigma}
\rfun{\mathrm{B}^{(i)}_{\nu}}{\sigma} \coloneqq
\begin{cases}
\mathrm{K}\text{ or }\rfun{\mathrm{I}_{\ii\nu}}{\sigma}
& \rbr{\mscrl, \mscrs\mscrv} = \rbr{+,+}, \\
\mathrm{F}\text{ or }\rfun{\mathrm{G}_{\ii\nu}}{\sigma}
& \rbr{\mscrl, \mscrs\mscrv} = \rbr{+,-}, \\
\mathrm{J}\text{ or }\rfun{\mathrm{Y}_{\nu}}{\sigma}
& \rbr{\mscrl, \mscrs\mscrv} = \rbr{-,+}, \\
\mathrm{K}\text{ or }\rfun{\mathrm{I}_{\nu}}{\sigma}
& \rbr{\mscrl, \mscrs\mscrv} = \rbr{-,-}. \\
\end{cases}
\end{align}
\item Adapted to imaginary order, $\rfun{\mathrm{F}_{\nu}}{\sigma}$ and
$\rfun{\mathrm{G}_{\nu}}{\sigma}$ are defined in \footfullcite{Dunster1990}.
\end{itemize}
\end{frame}

\begin{frame}
{Physical mode functions}%
{233}
\begin{itemize}
\item Physical mode functions are expected to be regular on the boundary.
\item $\rbr{+,+}$: $\vbr{\rfun{\mathrm{I}_{\ii\nu}}{\sigma}} \to +\infty$
as $\alpha \to +\infty$
\item $\rbr{-,+}$: 
\begin{itemize}
\item
$\forall n \in \BbbZ$, $\vbr{\rfun{\mathrm{Y}_{n}}{\sigma}} \to +\infty$ as 
$\alpha \to -\infty$.
\item
$\forall\nu\in\BbbR\backslash\BbbZ$, choose $\mathrm{J}_{-\nu}$ instead of 
$\mathrm{Y}_{+\nu}$, since $\mathrm{J}_{\pm\nu}$ are also linearly independent.
\item
$\forall\nu\in\BbbR^+\backslash\BbbZ$, $\vbr{\rfun{\mathrm{J}_{-\nu}}{\sigma}} 
\to +\infty$ as $\alpha \to -\infty$.
\end{itemize}
\item $\rbr{-,-}$: $\vbr{\rfun{\mathrm{K}_{\nu}}{\sigma}} \to +\infty$
as $\alpha \to -\infty$; $\vbr{\rfun{\mathrm{I}_{\nu}}{\sigma}} \to +\infty$
as $\alpha \to +\infty$
\item These are not to included in the space of physical wave functions.
\item For $\rbr{-,+}$, \alert{only $\mathrm{J}_{\nu}$ with 
$\nu \ge 0$ survives}.
\end{itemize}

\end{frame}


\begin{frame}%
{Matching quantum number with classical first integral}%
{233}
\begin{itemize}
\item Baustelle
\item In order to match the quantum number $k_\beta$ (or \alert{linearly}, 
$\nu$) with the classical first integral $p_\beta$, one may apply the 
\alert{principle of constructive interference}.

\item Writing the mode function in the WKB form
\begin{align}
\rfun{\Psi_{k_\beta}}{\beta,\chi} \sim \sqrt{\rho}\,\ee^{\ii S/\hslash},
\qquad S/\hslash \gg 1\text{ and }k_\beta \gg 1,
\end{align}
the principle demands that $\partial S/\partial k_\beta = 0$ be equivalent to 
the classical trajectory.

\end{itemize}
\end{frame}

\begin{frame}%
{Matching quantum number with classical first integral}%
{$\rbr{+,+}$ as exemplar}
\begin{itemize}
\item Fixing $\nu/\sigma>1$, the asymptotic expansion
reads
\begin{align}
\rfun{\mathrm{K}_{\ii \nu}}{\sigma} \sim
\frac{\sqrt{2\pp}\,
\rfun{\cos}{\sqrt{\nu^2-\sigma^2}-\nu\arccos\frac{\nu}{\sigma}-\frac{\pp}{4}}
}{
\rbr{\nu^2-\sigma^2}^{1/4}\ee^{\pp\nu/2}}
+\rfun{\Omicron}{\frac{1}{\sigma}}.
\end{align}

There are two phases with opposite signs. %$\rfun{\theta_\nu^{\pm}}{\sigma}$.
Applying the principle to the full mode function $\Psi_\nu$, one has
$\sigma/\nu = \rfun{\sech}{\mscrs_\beta\gamma}$, which match the trajectory if
\begin{align}
\hslash k_\beta \equiv \hslash \sqrt{\frac{3}{2\varkappa}}g\hslash\nu
= p_\beta,
\end{align}
as expected.

\item Fixing $\nu/\sigma<1$, the asymptotic expansion is not oscillatory,
but exponential; hence no WKB approximation works.

\item The conclusions also hold for $\rfun{\mathrm{F}_{\ii\nu}}{\sigma}$
and $\rfun{\mathrm{G}_{\ii\nu}}{\sigma}$ for $\rbr{+,-}$, as well as 
$\rfun{\mathrm{J}_{\nu}}{\sigma}$ for $\rbr{-,+}$.
\end{itemize}
\end{frame}

\begin{frame}%
{Inner product for wave functions}%
{Schrödinger product}
\begin{itemize}
\item To make sense of amplitude and wave packet, an inner product is necessary

\item A common starting point of most approaches is the \alert{Schrödinger 
product}
\begin{align}
\rbr{\Psi_1,\Psi_2}_\text{S} \coloneqq
\int \dd \chi\,\rfun{\Psi_1^*}{\beta,\chi} \rfun{\Psi_2}{\beta,\chi};
\end{align}

\item In terms of a norm, $\rbr{\Psi, \Psi}_\text{S} \equiv \int \dd \chi\,
\rfun{\rho_\text{S}}{\beta,\chi},$ in which $\rho_\text{S} \coloneqq 
\Psi^*\Psi$. Manifestly $\rho_\text{S} \ge 0$; one has $\rbr{\Psi, 
\Psi}_\text{S} > 0$.

\item
The corresponding Schrödinger current does not satisfy continuity equation
$\dot{\rho}_\text{S} + \nabla\cdot \vec{j}_\text{S} = 0$, because 
\cref{eq:WDW-10} is KG-like.
\item $\mathrm{K}_{\ii\nu}$%
\footfullcite{Yakubovich2006,Passian2009,Szmytkowski2010} for $\rbr{+,+}$, 
$\mathrm{F}_{\ii\nu}$ and $\mathrm{G}_{\ii\nu}$ (not $\mathrm{J}_{\ii\nu}$!) 
for $+,-$ can be proved to be orthogonal and complete individually, as well 
as normalised.
\end{itemize}
\end{frame}

\begin{frame}%
{Peculiarity for the phantom model}%
{Orthonormality and completeness for mode functions; self-adjointness for 
operators}
\begin{itemize}
% \item Imposing self-adjointness for $\widehat{p}_\chi = -\ii\hslash
% \partial_\chi$, one finds that $\mathrm{J}_{-\vbr{\nu}}$ leads to
% divergent $\rbr{\mathrm{J}_{-\vbr{\nu}}, \widehat{p}_\chi
% \mathrm{J}_{-\vbr{\nu}}}_\text{S}$, since\footfullcite[eq.\ (10.2.2)]%
% {NIST:DLMF}
% \begin{align}
% \rfun{\mathrm{J}_{\nu}}{z} = \rbr{\frac{z}{2}}^\nu
% \rbr{\frac{1}{\rfun{\mupGamma}{\nu+1}} + \rfun{\Omicron}{z^2}}.
% \end{align}

\item $\cbr{\rfun{\mathrm{J}_{\nu}}{\sigma}}$'s are not orthogonal
\begin{align}
\int_0^{+\infty}\frac{\dd\sigma}{\sigma}
\rfun{\mathrm{J}_{\nu}}{\sigma}
\rfun{\mathrm{J}_{\tilde{\nu}}}{\sigma} =
\frac{2\rfun{\sin}{\pp\rbr{\nu-\tilde{\nu}}/2}}{\pp\rbr{\nu^2-\tilde{\nu}^2}}.
\end{align}


\item Imposing self-adjointness for $\widehat{p}_\chi^2$, one finds that
\end{itemize}
\end{frame}



\section*{Appendix}

\begin{frame}%
{Integration of the transformed first Friedmann equation}%
{$p_\beta \neq 0$}
In order to integrate the equation under the implicitising gauge
\begin{align}
\mscrs\frac{p_\beta^2}{12}\rbr{
-\mscrl\frac{\varkappa^{1/2}}{6}\rbr{\frde{\chi}{\beta}}^2
+\varkappa^{-1/2}}
-\varkappa^{3/2}V\ee^{g\mscrs_\chi\chi} = 0,
\end{align}
one can substitute
\begin{align}
\gamma \coloneqq \sqrt{\frac{3}{2\varkappa}}g\beta, \qquad
\widetilde{\sigma}^2 \coloneqq \frac{p_\beta^2}{12\varkappa^2 \vbr{V}}
\ee^{-g\mscrs_\chi\chi},
\end{align}
to get
\begin{align}
\rbr{\frde{\widetilde{\sigma}}{\gamma}}^2
+\mscrl\rbr{\mscrs\mscrv - \widetilde{\sigma}^2} = 0
\quad\Rightarrow\quad
\frde{\gamma}{\widetilde{\sigma}} = \pm
\frac{1}{\sqrt{\mscrl\rbr{- \mscrs\mscrv + \widetilde{\sigma}^2}}},
\end{align}
which is of the standard inverse hyperbolic / trigonometric form for
$\rbr{+,+}$, $\rbr{+,-}$ and $\rbr{-,+}$.
\end{frame}



\begin{frame}%
{Integration of the separated mss.\ Wheeler--DeWitt equation}%
{233}
In order to integrate the separated minisuperspace Wheeler--DeWitt equation
\begin{align}
\ee^{-\ii k_\beta\mscrs_\beta\beta}\rbr{
-\mscrl\mscrs\frac{\hslash^2}{2\varkappa^{3/2}}\rfun{\psi''}{\chi}
-\mscrs\frac{\hslash^2k_\beta^2}{12\varkappa^{1/2}}\rfun{\psi}{\chi}
+\varkappa^{3/2}V\ee^{g\mscrs_\chi\chi}\rfun{\psi}{\chi}
} = 0,
\end{align}
one can transform
\begin{align}
\nu &\coloneqq \sqrt{\frac{2\varkappa}{3}}\frac{k_\beta}{g},\qquad
\sigma^2 \coloneqq 
\frac{8\varkappa^3\vbr{V}\ee^{g\mscrs_\chi\chi}}{\hslash^2 g^2},
\tag{\ref{eq:trsf-quantum-sigma} rev.}
\end{align}
to get
\begin{align}
\sigma^2\rfun{\psi''}{\sigma}
+\sigma\rfun{\psi'}{\sigma}
+\mscrl\rbr{\nu^2-\mscrs\mscrv\sigma^2}\rfun{\psi}{\sigma} = 0,
\end{align}
which is of the standard Besselian form.
\end{frame}

\begin{itemize}
\item
\end{itemize}

\begin{frame}
  \frametitle{Allgemeines}

  \begin{itemize}
  \item Mit diesem \emph{beamer theme} ist es möglich, Präsentationen in
    \LaTeX{} mit der Beamer-Klasse zu erstellen, die dem Corporate Design der
    Universität zu Köln entsprechen
  \item Auf die Beamer-Klasse wird in diesem Dokument nicht näher eingegangen,
    nähere Informationen finden Sie unter
    \url{http://latex-beamer.sourceforge.net/}
  \end{itemize}

\end{frame}

\begin{frame}[fragile]
  \frametitle{Laden des Themes}
  \begin{block}{Das Theme kann mit den folgenden Optionen geladen werden}
    \begin{small}
\begin{verbatim}
\usetheme[%
% uk,      %% Farben aller Fakultaeten
wiso,     %% Wiso-Fakultaet
% jura,    %% Rechtswissenschaftliche Fakultaet
% medizin, %% Medizinische Fakultaet
% philo,   %% Philosophische Fakultaet
% matnat,  %% Mathematisch-Naturwissenschaftliche Fakultaet
% human,   %% Humanwissenschaftliche Fakultaet
% verw,    %% Universitaetsverwaltung
]{UzK}
\end{verbatim}
    \end{small}

  \end{block}
\end{frame}

\begin{frame}
  \frametitle{Die Fußzeile}

  \begin{itemize}
  \item Es stehen verschiedene Fußzeilen zur Auswahl, die als Option
    beim Laden des \emph{themes} übergeben werden:
    \begin{itemize}
    \item Balken mit allen Fakultätsfarben (Option \texttt{uk})
    \item Balken in jeweils einer Fakultätsfarbe (Optionen \texttt{wiso, jura,
        medizin, philo, matnat, human, verw})\footnote{Es werden die offiziellen
        RGB-Werte aus dem 2-D Handbuch Corporate Design verwendet.}
    \end{itemize}
  \item "`Universität zu Köln"' sowie der Name der Fakultät sind im
    Theme definiert, das Institut oder Seminar kann mit dem Befehl
    \texttt{\textbackslash institute\{\}} festgelegt werden
  \item Die Optionen sind im Quellcode dieser Präsentation dokumentiert
  \end{itemize}

\end{frame}

\begin{frame}
  \frametitle{Englische Präsentationen}
  \begin{itemize}
  \item Der Universitäts- sowie die Fakultätsnamen werden
    standardmäßig auf Deutsch angezeigt.
  \item Übergeben Sie dem Paket \texttt{babel} die Option
    \texttt{english}, so werden diese Namen entsprechen angepasst.
  \item Die Übersetzungen können in der Theme-Datei
    \texttt{beamerthemeUzK.sty} geändert werden
  \end{itemize}

\end{frame}

\begin{frame}
  \frametitle{\texttt{block}-Umgebungen}
  \begin{block}{Standard (\texttt{block})}
    Verwendet die Farbe "`Blaugrau Mittel"' als Blocktitel-Hintergrund
  \end{block}

  \begin{exampleblock}{\texttt{exampleblock}}
    Bei Verwendung der Fußzeile mit allen Fakultätsfarben
    Titelhintergrund in Wiso-Grün, sonst in der jeweiligen
    Fakultätsfarbe
  \end{exampleblock}

  \begin{alertblock}{\texttt{alertblock}}
    Verwendet das Rot der Folientitel
  \end{alertblock}

\end{frame}


\begin{frame}
  \frametitle{Installation}
  \begin{itemize}
  \item Das Theme besteht aus den Dateien
    \texttt{beamerthemeUzK.sty} und \texttt{beamercolorthemeUzK.sty}
    sowie den Grafikdateien \texttt{logo.pdf} und
    \texttt{logo-small.pdf}.
  \item Das Theme kann auf zwei Arten verwendet werden:
    \begin{enumerate}
    \item Die vier Dateien werden in den selben Ordner wie die zu
      erstellende Präsentation gelegt
    \item Die vier Dateien werden im lokalen \emph{texmf}-Baum abgelegt
    \end{enumerate}
  \item Die zweite Variante ist der ersten vorzuziehen, da das Theme
    so an einem zentralen Ort vorliegt
  \end{itemize}
\end{frame}


\begin{frame}
  \frametitle{ToDo}

  \begin{block}{Was noch zu tun ist\ldots}
    \begin{itemize}
    \item Erstellen einer eigenen Titelseite
    \item \ldots
    \end{itemize}
  \end{block}

\end{frame}

\end{document}
