%% UzK - A BEAMER THEME FOR THE UNIVERSITY OF COLOGNE
%% http://solstice.github.com/uzk-theme/

\documentclass{beamer}

% Multilingual support
\usepackage{polyglossia}

% more symbols
\usepackage{textcomp}

% fontspec, realscripts, metalogo fot XeLaTeX
\usepackage{xltxtra}
	% Unicode fonts
	\setmainfont{CMU Serif}
	\setsansfont{CMU Sans Serif}
	\setmonofont{CMU Typewriter Text}
\usepackage{amsfonts}
\usepackage{unicode-math}
	\setmathfont{Latin Modern Math} % default
	\setmathfont{Latin Modern Math}[range={
		"1D608-"1D63B},	% Italic sans serif, Latin, uppercase
		sans-style=italic]
	\setmathfont{Latin Modern Math}[range={
		"1D6E2-"1D6FA,	% Italic Greek, uppercase
		"1D6FC-"1D71B},	% Italic Greek, lowercase
		math-style=ISO]
	\setmathfont{Latin Modern Math}[range={
		"00391-"003A9,	% Upright Greek, uppercase
		"003B1-"003F5,	% Upright Greek, lowercase
		"1D6A8-"1D6E1},	% Bold Greek
		sans-style=upright]
	\setmathfont{Asana Math}[range={
		\mathbin}] %\mathord
	\setmathfont{STIX Math}[range={
		"02609}] % ☉
	\setmathfont{XITS Math}[range={
		"1D4B6-"1D4CF}] % Script, Latin, lowercase
	%\setmathfont{⟨font name⟩}[range=⟨unicode range⟩,⟨font features⟩]

\usepackage{xeCJK}
\setmainlanguage{english}
\setotherlanguage{ngerman}

\usefonttheme{professionalfonts}
%  don't change fonts inside beamer

% AMS--related
\usepackage{amsmath,amssymb}

% ':=' as \coloneqq
\usepackage{mathtools}
\usepackage{siunitx}
\usepackage{cleveref}

% some unicode characters
% ≙ for equal with hat


% Mathematical constants
\newcommand{\ii}{{\Bbbi}}
\newcommand{\ee}{{\Bbbe}}
\newcommand{\pp}{{\Bbbpi}}

% Math operators
\DeclareMathOperator{\arcsinh}{arcsinh}
\DeclareMathOperator{\arccosh}{arccosh}
\DeclareMathOperator{\arctanh}{arctanh}
\DeclareMathOperator{\arccoth}{arccoth}
\DeclareMathOperator{\arcctgh}{arcctgh}
\DeclareMathOperator{\arcsech}{arcsech}
\DeclareMathOperator{\arccsch}{arccsch}

\DeclareMathOperator{\BesselJ}{J}
\DeclareMathOperator{\BesselY}{Y}
\DeclareMathOperator{\BesselF}{F}
\DeclareMathOperator{\BesselG}{G}
\DeclareMathOperator{\BesselI}{I}
\DeclareMathOperator{\BesselK}{K}
\DeclareMathOperator{\BesselL}{L}

\DeclareMathOperator{\sgn}{sgn}
\DeclareMathOperator{\grad}{grad}
\DeclareMathOperator{\curl}{curl}
\DeclareMathOperator{\rot}{rot}
\DeclareMathOperator{\opdiv}{div}
\DeclareMathOperator{\opdeg}{deg}

\DeclareMathOperator{\sech}{sech}
\DeclareMathOperator{\csch}{csch}

\DeclareMathOperator{\diag}{diag}
\DeclareMathOperator{\tr}{tr}
\DeclareMathOperator{\rank}{rank}

\DeclareMathOperator{\ad}{ad}

\DeclareMathOperator{\expi}{expi}

% Differentials
\newcommand{\DD}{\BbbD}
\newcommand{\dd}{\Bbbd}
\newcommand{\dva}{\mupdelta} % no better way?!
\newcommand{\Dva}{\mupDelta}

% Equal marks
\newcommand{\eeq}{{\overset{!}{=}}}
\newcommand{\lls}{{\overset{!}{<}}}
\newcommand{\ggt}{{\overset{!}{>}}}
\newcommand{\lle}{{\overset{!}{\le}}}
\newcommand{\gge}{{\overset{!}{\ge}}}

% Group and Algebras
\newcommand{\SO}{\msansS\msansO}
\newcommand{\SU}{\msansS\msansU}
\newcommand{\so}{\mfraks\mfrako}
\newcommand{\su}{\mfraks\mfraku}
% Bracket-like
\newcommand{\rbr}[1]{{\left(#1\right)}}
\newcommand{\sbr}[1]{{\left[#1\right]}}
\newcommand{\cbr}[1]{{\left\{#1\right\}}}
\newcommand{\abr}[1]{{\left<#1\right>}}
\newcommand{\vbr}[1]{{\left|#1\right|}}
\newcommand{\dvbr}[1]{{\left\|#1\right\|}}
\newcommand{\fat}[2]{{\left.#1\right|_{#2}}}
% Functions; note the space between the name and the bracket!
\newcommand{\rfun}[2]{{#1}\mathopen{}\left(#2\right)\mathclose{}}
\newcommand{\sfun}[2]{{#1}\mathopen{}\left[#2\right]\mathclose{}}
\newcommand{\cfun}[2]{{#1}\mathopen{}\left\{#2\right\}\mathclose{}}
\newcommand{\afun}[2]{{#1}\mathopen{}\left<#2\right>\mathclose{}}
\newcommand{\vfun}[2]{{#1}\mathopen{}\left|#2\right|\mathclose{}}
% Fraction-like
\newcommand{\frde}[2]{{\frac{\dd{#1}}{\dd{#2}}}}
\newcommand{\frDe}[2]{{\frac{\DD{#1}}{\DD{#2}}}}
\newcommand{\frpa}[2]{{\frac{\partial{#1}}{\partial{#2}}}}
\newcommand{\frdva}[2]{{\frac{\dva{#1}}{\dva{#2}}}}

% overline-like marks
\newcommand{\ol}[1]{{\overline{{#1}}}}
\newcommand{\ul}[1]{{\underline{{#1}}}}
\newcommand{\tld}[1]{{\widetilde{{#1}}}}
\newcommand{\ora}[1]{{\overrightarrow{#1}}}
\newcommand{\ola}[1]{{\overleftarrow{#1}}}
\newcommand{\td}[1]{{\widetilde{#1}}}
\newcommand{\what}[1]{{\widehat{#1}}}
%\newcommand{\prm}{{\symbol{"2032}}}

% Physical constants and parameters
\newcommand{\lc}{\mitsansc} % speed of light in vacuum
\newcommand{\bk}{\mitsansk} % Boltzmann's constant
\newcommand{\phs}{\hslash} % reduced Planck constant
\newcommand{\ph}{\Planckconst} % Planck constant

\newcommand{\nG}{\mitsansG} % Newton's constant
\newcommand{\aN}{\mitsansN} % Avogadro number
\newcommand{\ec}{\mitsanse} % unit electric charge

\newcommand{\gR}{\mitsansR} % gas constant

\newcommand{\plm}{m_\text{P}} % Planck mass
\newcommand{\pll}{l_\text{P}} % Planck length
\newcommand{\plt}{t_\text{P}} % Planck time

\newcommand{\hH}{\mitsansH} % Hubble parameter H
\newcommand{\hh}{\mitsansh} % Hubble parameter h
\newcommand{\dq}{\mitsansq} % Deceleration parameter q

\newcommand{\apE}{\alpha_\text{E}} % EM fine struct const
\newcommand{\apG}{\alpha_\text{G}} % Grav fine struct const

% Common symbols
\newcommand{\Ld}{\mscrL} % Lagrangian density
\newcommand{\fp}{p_\text{F}} % Fermi momentum
\newcommand{\fE}{\mscrE_\text{F}} % Fermi energy

% Others
\newcommand{\rSch}{R_\text{S}} % Schwarzschild radius

\newcommand{\fHor}{{\mscrh^+}} % future horizon
\newcommand{\pHor}{{\mscrh^-}} % past horizon

% Chemical elements
\usepackage[version=4]{mhchem}


% siunitx
% Astronomy
\DeclareSIUnit\parsec{pc}
\DeclareSIUnit\lightyear{ly}


%% Falls Anzeige der \sections, \subsections etc. gewuenscht, kann zB.
%% das infolines theme geladen werden. Wichtig ist jedoch, dass andere
%% Themes _vor_ dem UzK-Theme geladen werden.
%\useoutertheme{infolines}

%% Falls keine der Optionen zur Bestimmung der Fusszeile uebergeben werden    %%
%% werden alle Fakultaetsfarben verwendet. ---------------------------------- %%
\usetheme[%
%wiso,        %% Wiso-Fakultaet
%jura,        %% Rechtswissenschaftliche Fakultaet
%medizin,     %% Medizinische Fakultaet
%philo,       %% Philosophische Fakultaet
%matnat,      %% Mathematisch-Naturwissenschaftliche Fakultaet
%human,       %% Humanwissenschaftliche Fakultaet
%verw,        %% Universitaetsverwaltung
%nav,         %% Schaltet die Navigationssymbole ein
latexfonts,  %% Verwendet die latex-beamer-Standardschrift
%colorful,    %% Farbige Balken im infolines-Theme
%squares,     %% Aufzaehlungspunkte rechteckig
%nologo,      %% Kein Logo im Seitenhintergrund
]{UzK}

\title{Integrable Cosmological Models with Liouville Scalar Fields}

\author[Andrianov \and Lan \and Novikov \and \emph{Wang}]{
	Alexander A. Andrianov\inst{1,4} %\thanks{a.andrianov@spbu.ru}
	\and
	Chen Lan\inst{2} %\thanks{stlanchen@yandex.ru}
	\and
	Oleg O. Novikov\inst{1} %\thanks{o.novikov@spbu.ru}
	\and 
	\emph{Yi-Fan Wang}\inst{3}} %\thanks{yfwang@thp.uni-koeln.de}
%{David Kusterer\thanks{ 
%\href{mailto:kusterer@uni-koeln.de}{kusterer@uni-koeln.de} }%
%  \and%
%  Bernd 
%Weiß\thanks{\href{mailto:bernd.weiss@wiso.uni-koeln.de}{
%bernd.weiss@wiso.uni-koeln.de}}}

\institute[Forschungsinstitut für Soziologie]{%
Forschungsinstitut für Soziologie \\
Greinstraße 2\\
50939 Köln}

\institute[SPBU \and ELI-ALPS \and UzK \and UB]{
\inst{1} Saint-Petersburg State University, St.\ Petersburg 198504, Russia \and
\inst{2} ELI-ALPS, ELI-Hu NKft, Dugonics t\'er 13, Szeged 6720, Hungary \and
\inst{3} Institut f\"ur Theoretische Physik, Universit\"at zu K\"oln,
Z\"ulpicher Stra\ss e 77, 50937 K\"oln, Germany \and
\inst{4}
Institut de Ci\`encies del Cosmos (ICCUB), Universitat de Barcelona, Spain}


\begin{document}

\begin{frame}%[titlepage]
  \titlepage
\end{frame}

\begin{frame}{Outline}
  \tableofcontents
  % You might wish to add the option [pausesections]
\end{frame}


\section{Introduction}

\section{Classical model}

\begin{itemize}
\item Flat Robertson--Walker metric
$\dd s^2 = -\rfun{N^2}{t}\,\dd t^2
+\varkappa^{-1/2}\ee^{2\rfun{\alpha}{t}}\,\dd\Omega_3^2$,
where $\varkappa = 8\pp\nG$, $\dd\Omega_3^2$ dimensionless spacial 
metric
\item Homogeneous real Klein--Gordon with potential
(dubbed Liouville) $V\ee^{\lambda\phi}$ , $\lambda, V\in \BbbR$.

\item Total action $\mscrS = S_{\text{EH}} + S_\text{GHY} + S_\text{L}
= \int\dd\Omega_3^2\int\dd t\,L$,
\begin{align}
%S_{\text{Einstein--Hilbert}} + S_\text{Gibbons--Hawking--York} + 
%S_\text{Liouville}
L&\coloneqq \varkappa^{3/2}N\ee^{3\alpha}
\rbr{-\frac{3}{\varkappa}\frac{\dot{\alpha}^2}{N^2}
+\mscrl\frac{\dot{\phi}^2}{2N^2}-V\ee^{\lambda\phi}},
\end{align}
in which dot means $\dd/\dd t$, $\mscrl = \pm 1$ corresponds to quintessence
/ phantom model, respectively.

\item Choosing $\overline{N} \coloneqq N\ee^{-3\alpha}$, the effective
Lagrangian transforms to
\begin{align}
L_\text{e} = \varkappa^{3/2}\overline{N}
\rbr{-\frac{3}{\varkappa}\frac{\dot{\alpha}^2}{\overline{N}^2}
+\mscrl\frac{\dot{\phi}^2}{2\overline{N}^2} - V\ee^{\lambda\phi+6\alpha}}
\end{align}

\item Defining $\Delta \coloneqq \lambda^2 - 6\mscrl\varkappa$,
$\mscrs \coloneqq \sgn \Delta$ and
$g \coloneqq \mscrs \sqrt{\vbr{\Delta}} \equiv \mscrs\sqrt{\mscrs\Delta}$,
the rescaled special orthogonal transformation
\begin{align}
\begin{pmatrix}
\alpha \\ \phi
\end{pmatrix} = \frac{\mscrs}{g}
\begin{pmatrix}
\lambda & -\mscrl\kappa \\
-6 & \lambda
\end{pmatrix}
\begin{pmatrix}
\mscrs_\beta \beta \\ \mscrs_\chi \chi
\end{pmatrix}\qquad\text{where } \mscrs_\beta, \mscrs_\chi = \pm 1
\end{align}
gives the decoupled Lagrangian
\begin{align}
L_\text{d} = \varkappa^{3/2}\overline{N}
\rbr{-\mscrs\frac{3}{\varkappa}\frac{\dot{\beta}^2}{\overline{N}^2}
+\mscrl\mscrs\frac{\dot{\chi}^2}{2\overline{N}^2}
-V\ee^{\mscrs_\chi g\chi}},
\label{eq:lagrangian-decoupled}
\end{align}

\item Since $\beta$ is cyclic in \cref{eq:lagrangian-decoupled},
the second Friedmann equation can be integrated
\begin{align}
\text{const.} \equiv p_\beta \coloneqq \frpa{L_\text{d}}{\dot{\beta}} &=
-6\mscrs\varkappa^{1/2}\frac{\dot{\beta}}{\overline{N}} \\
&=
-6\mscrs\mscrs_\beta\frac{\varkappa^{1/2}}{g}
\frac{\lambda\dot{\alpha}+\mscrl\varkappa\dot{\phi}}{\overline{N}}.
\end{align}

\item Taking the gauge $\overline{N} = -6\mscrs\sqrt{\varkappa}\dot{\beta}/p_\beta$, the first Friedmann equation can be integrated
\begin{align}
\ee^{6\alpha+\lambda\phi} \equiv \ee^{g\mscrs_\chi\chi} =
\frac{p_\beta^2}{12\varkappa^2\vbr{V}}
\rfun{f^2}{\sqrt{\frac{3}{2\varkappa}}
\rbr{\alpha\lambda+\mscrl\varkappa\phi}},
\\
\rfun{f}{\gamma} \coloneqq
\begin{cases}
\rfun{\cosh}{\gamma+C} & \mscrl = +1, \mscrs\mscrv = +1, \\
\rfun{\sinh}{\gamma+C} & \mscrl = +1, \mscrs\mscrv = -1, \\
\rfun{\cos}{\gamma+C} & \mscrl = -1, \mscrs\mscrv = +1, \\
\ii\rfun{\sin}{\gamma+C} & \mscrl = -1, \mscrs\mscrv = -1, \\
\end{cases}
\end{align}
in which $\mscrv \coloneqq \sgn V$.

\end{itemize}



\section{Quantum model with constant potential}

\section{Classical model with exponential potential}

\section{Quantum model with exponential potential}

\section{Wave packets and their matching}

\begin{frame}
  \frametitle{Allgemeines}

  \begin{itemize}
  \item Mit diesem \emph{beamer theme} ist es möglich, Präsentationen in
    \LaTeX{} mit der Beamer-Klasse zu erstellen, die dem Corporate Design der
    Universität zu Köln entsprechen
  \item Auf die Beamer-Klasse wird in diesem Dokument nicht näher eingegangen,
    nähere Informationen finden Sie unter
    \url{http://latex-beamer.sourceforge.net/}
  \end{itemize}

\end{frame}

\begin{frame}[fragile]
  \frametitle{Laden des Themes}
  \begin{block}{Das Theme kann mit den folgenden Optionen geladen werden}
    \begin{small}
\begin{verbatim}
\usetheme[%
% uk,      %% Farben aller Fakultaeten
wiso,     %% Wiso-Fakultaet
% jura,    %% Rechtswissenschaftliche Fakultaet
% medizin, %% Medizinische Fakultaet
% philo,   %% Philosophische Fakultaet
% matnat,  %% Mathematisch-Naturwissenschaftliche Fakultaet
% human,   %% Humanwissenschaftliche Fakultaet
% verw,    %% Universitaetsverwaltung
]{UzK}
\end{verbatim}
    \end{small}

  \end{block}
\end{frame}

\begin{frame}
  \frametitle{Die Fußzeile}

  \begin{itemize}
  \item Es stehen verschiedene Fußzeilen zur Auswahl, die als Option
    beim Laden des \emph{themes} übergeben werden:
    \begin{itemize}
    \item Balken mit allen Fakultätsfarben (Option \texttt{uk})
    \item Balken in jeweils einer Fakultätsfarbe (Optionen \texttt{wiso, jura,
        medizin, philo, matnat, human, verw})\footnote{Es werden die offiziellen
        RGB-Werte aus dem 2-D Handbuch Corporate Design verwendet.}
    \end{itemize}
  \item "`Universität zu Köln"' sowie der Name der Fakultät sind im
    Theme definiert, das Institut oder Seminar kann mit dem Befehl
    \texttt{\textbackslash institute\{\}} festgelegt werden
  \item Die Optionen sind im Quellcode dieser Präsentation dokumentiert
  \end{itemize}

\end{frame}

\begin{frame}
  \frametitle{Englische Präsentationen}
  \begin{itemize}
  \item Der Universitäts- sowie die Fakultätsnamen werden
    standardmäßig auf Deutsch angezeigt.
  \item Übergeben Sie dem Paket \texttt{babel} die Option
    \texttt{english}, so werden diese Namen entsprechen angepasst.
  \item Die Übersetzungen können in der Theme-Datei
    \texttt{beamerthemeUzK.sty} geändert werden
  \end{itemize}

\end{frame}

\begin{frame}
  \frametitle{\texttt{block}-Umgebungen}
  \begin{block}{Standard (\texttt{block})}
    Verwendet die Farbe "`Blaugrau Mittel"' als Blocktitel-Hintergrund
  \end{block}

  \begin{exampleblock}{\texttt{exampleblock}}
    Bei Verwendung der Fußzeile mit allen Fakultätsfarben
    Titelhintergrund in Wiso-Grün, sonst in der jeweiligen
    Fakultätsfarbe
  \end{exampleblock}

  \begin{alertblock}{\texttt{alertblock}}
    Verwendet das Rot der Folientitel
  \end{alertblock}

\end{frame}


\begin{frame}
  \frametitle{Installation}
  \begin{itemize}
  \item Das Theme besteht aus den Dateien
    \texttt{beamerthemeUzK.sty} und \texttt{beamercolorthemeUzK.sty}
    sowie den Grafikdateien \texttt{logo.pdf} und
    \texttt{logo-small.pdf}.
  \item Das Theme kann auf zwei Arten verwendet werden:
    \begin{enumerate}
    \item Die vier Dateien werden in den selben Ordner wie die zu
      erstellende Präsentation gelegt
    \item Die vier Dateien werden im lokalen \emph{texmf}-Baum abgelegt
    \end{enumerate}
  \item Die zweite Variante ist der ersten vorzuziehen, da das Theme
    so an einem zentralen Ort vorliegt
  \end{itemize}
\end{frame}


\begin{frame}
  \frametitle{ToDo}

  \begin{block}{Was noch zu tun ist\ldots}
    \begin{itemize}
    \item Erstellen einer eigenen Titelseite
    \item \ldots
    \end{itemize}
  \end{block}

\end{frame}

\end{document}
